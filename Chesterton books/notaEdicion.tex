\chapter{Nota a la primera edición}

\textbf{Esta edición esta basada} en una versión digital del libro,
publicada en Argentina en 1998,
de la editorial Porrúa y presenta muchas mejoras respecto a dicha versión.

Se han añadido referencias históricas y diversas notas al margen, para
ayudar al lector dando contexto y significado a ciertos pasajes del libro.

Esta edición esta motivada por la carencia de una versión digital de calidad
y disponible a todo el público. La única versión parecida, aunque en ingles,
es la distribuida por el Proyecto Gutenberg.\nota{Este proyecto se encarga de digitalizar libros antiguos de forma gratuita, se pueden encontrar en: \url{https://www.gutenberg.org/}}

% El lector hará bien en nota que, si esta usando la versión digital de libro, las referencias a las notas al final (como la del proyecto Gutenberg en el párrafo anterior) son enlaces. Más aún, el número de la nota al final es otro enlace para regresar al texto: así se podrá reducir el tedio de buscar en cual página estaba leyendo.

\section*{Colofón}

La fuente principal es Latin Modern Roman a 10pt, Questa Grande para los títulos de los capítulos, Iwona Condensada para las notas al margen e Iosevka para los enlaces a internet y demás contenido usualmente mono-espaciado.
