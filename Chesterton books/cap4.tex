\chapter{La ética del país de los elfos}%
\label{cha:La ética en el país de los elfos}

\textsc{cuando el hombre de negocios} reprocha al joven empleado su idealismo, por lo general lo hace en
estos términos: \textquote{¡Ah! sí, cuando se es joven, se tienen esos ideales abstractos y esos castillos en el aire;
pero llegando a la madurez, todos se desvanecen como nubes y se empieza a creer en la política práctica,
a usar de los medios de que se dispone y a reconciliarse con el mundo tal cual es}. Por lo menos cuando
yo era niño así me hablaban hombres filántropos y venerables que hoy yacen en sus honradas tumbas.

Pero desde entonces he crecido y he descubierto que esos viejos filántropos me mentían.

Que en realidad sucedió lo contrario de lo que ellos me decían que iba a suceder. Decían que
perdería mis ideales y comenzaría a creer en los métodos prácticos de la política.

Y no he perdido en absoluto mis ideales; mi fe es fundamentalmente exacta a lo que ha sido
siempre. Lo que he perdido es mi antigua infantil confianza en la política práctica. Continúo tan
interesado como antes en la Batalla de Armaguedón; pero no estoy tan interesado en las Elecciones
Generales. Cuando era bebé, a su sola mención saltaba en las rodillas de mi madre. No; la visión es
siempre sólida y fidedigna; la visión siempre es un hecho. La realidad es lo que con frecuencia resulta un
fraude.

Creo en el liberalismo tanto como siempre; más que nunca. Pero en una época rosada de inocencia,
creí en los liberales.

Teniendo que trazar ahora el curso de mi especulación personal, tomo este ejemplo de una de las
creencias que persisten, porque tal vez se la pueda contar como única tendencia positiva. Crecí como
liberal y he creído siempre en la democracia, en la doctrina elemental de una humanidad autogobernada.

Si alguno encuentra esta frase vaga o confusa, sólo puedo detenerme un momento para explicar que
el principio democrático, según yo lo entiendo, podría enunciarse en dos proposiciones. La primera es
ésta: que las cosas comunes a todos los hombres, son más importantes que las cosas peculiares de
cualquier hombre. Las cosas ordinarias tienen más valor que las extraordinarias; aún mejor: son más
extraordinarias. El hombre, es algo más imponente que los hombres; algo más sorprendente. El sentido
milagroso de lo humano en sí, debe ser siempre algo más vívido para nosotros que todas las maravillas
del poder, de la inteligencia, del arte o de la civilización.

El vulgar hombre sobre sus dos piernas, como tal, debería ser sentido como algo más emocionante
que cualquier música, más sorprendente que cualquier caricatura. Morir es más trágico que morir de
hambre.

Tener nariz, es más cómico aún que tener una nariz normanda.

Ese, es el primer principio demócrata: que las cosas esenciales para los hombres, son las cosas que
poseen en común; no las cosas que poseen por separado. Y el segundo principio es sencillamente este:
que el instinto o deseo político, es una de esas cosas que poseen en común.

Enamorarse es más poético que languidecer en poesías.

La idea demócrata es que el gobierno (ayudando a regir un país) es algo así como enamorarse y no
como languidecer en poesías. No algo análogo a tocar el órgano en una iglesia, a pintar telas o a descubrir
el Polo Norte ( ¡pérfida costumbre!) y a hechos por el estilo. Porque no deseamos que los hombres hagan
esas cosas, a no ser que las hagan muy bien. Por el contrario, la idea demócrata es que el gobierno es algo
análogo a escribir las propias cartas de amor, o a sonarse la nariz. Estas cosas deseamos cine los hombres
las hagan por sí mismos, aunque las hagan muy mal. No discuto aquí la exactitud de ninguna de estas
concepciones; sé que algunos modernos andan por ahí pidiendo que los científicos les elijan sus esposas,
y por lo que sabemos, pronto pedirán que las niñeras les suenen la nariz.

Digo solamente, que la humanidad reconoce estas universales funciones humanas, y que la
democracia incluye al gobierno entre ellas. Abreviando, la teoría demócrata es ésta: que las cosas más
terriblemente importantes deben dejarse libradas al hombre, la complementación de los sexos, la
educación de la juventud, las leyes del estado. Esto es democracia: yen esto es en lo que he creído
siempre.

Pero desde mi juventud hasta hoy, hay algo que nunca pude comprender. Nunca he podido
comprender de dónde es que la gente ha sacado la idea, de que la democracia se opone en cierta forma a
la tradición.

Evidentemente, la tradición es sólo la democracia prolongada a través del tiempo. Es creer en un
concierto de vulgares voces humanas, más que en un registro aislado y arbitrario de los hechos. El
hombre que cita a un historiador alemán en su ataque a la tradición de la Iglesia Católica, apela
estrictamente a la aristocracia.

Recurre a la superioridad de un experto para oponerla a la tremenda autoridad de una muchedumbre
popular.

Es fácil ver por qué una leyenda es tratada, y debe ser tratada, con más respeto que un libro de
historia.

La leyenda, generalmente la hace la mayoría (le la gente sensata de un pueblo. El libro,
generalmente está escrito por un sólo loco del pueblo. Aquellos que contra la tradición arguyen que los
hombres de ayer eran ignorantes, pueden ir con sus argumentos al Club Carlton,\nota{Club aristocrático de Londres.} manifestando que los
votantes de los garitos son ignorantes. No nos hace nada.

Si cuando se trata de asuntos cotidianos, concedemos gran importancia a la opinión unánime del
común de los hombres, no hay razón para que la menospreciemos cuando se trata de fábulas y de historia.

La tradición podría definirse como una extensión de esa franquicia.

Tradición, significa dar votos a la más oscurecida de todas las clases: nuestros antecesores. Es la
democracia de los muertos. La tradición rehúsa someterse a la pequeña y arrogante oligarquía de aquellos
que casualmente, andan por ahí.

La democracia pone objeciones a los hombres por ser incapacitados por el accidente de su
nacimiento; la tradición se las pone por ser incapacitados por el accidente de su muerte. La democracia
nos aconseja no desoír la opinión de un hombre bueno; aunque sea nuestro mucamo. La tradición nos
pide que no desoigamos la opinión de un hombre bueno; aunque sea nuestro padre. Yo por lo menos, no
puedo separar las ideas de democracia y de tradición; me parece evidente que ambas son una misma idea.

Tendremos a los muertos en nuestros concilios. Los antiguos griegos votaban en piedras; éstos,
votarán en lápidas. Todo es perfectamente oficial y correcto, puesto que muchas lápidas, como muchas
papeletas de votar, están marcadas con una cruz.

Debo decir primero, qué si he tenido una inclinación, siempre fue una inclinación a favor de la
democracia, y por consiguiente, de la tradición. Antes de llegar a ningún principio teórico o lógico, me
conformo con permitirme esta confesión personal: siempre he estado más inclinado a creer en el clamor
de la clase trabajadora, que a creer en esa selecta y perturbada clase literata, a la cual pertenezco. Prefiero
aún las fantasías y los prejuicios del pueblo que ve la vida desde dentro, a las demostraciones más claras
del pueblo que vé la vida desde fuera. Siempre creeré más en las fábulas de las viejas mujeres que en los
hechos de las viejas solteronas. Mientras la fantasía sea fantasía innata, puede ir tan lejos como le plazca.

Ahora, tengo que explicar una posición general y no pretendo estar entrenado para esas cosas. Por
consiguiente, me propongo hacerlo, escribiendo sucesivamente las tres o cuatro ideas fundamentales que
hallé por mí mismo y fielmente en la forma en que las hallé.

Luego, rápidamente he d sintetizarlas., agregando mi filosofía propia o religión natural; luego,
describiré mi sorprendente descubrimiento de que el todo, ya había sido descubierto. Había sido
descubierto por el Cristianismo. Pero de estas profundas persuasiones de las que debo dar cuenta
ordenadamente, la primera concernía a este elemento de tradición popular.

Y sin la explicación en curso, referente a la tradición y a la democracia, difícilmente podría exponer
con claridad mi experiencia mental. Y aún así, no sé si podré ser claro. Y ya me propongo por lo menos,
intentarlo.

Mi primera y última filosofía, aquella en la cual creo con fe inquebrantable, la aprendí en la nursery.

Y vagamente, la aprendí de una niñera; es decir, de la solemne y estrellada sacerdotiza de la
democracia y de la tradición. Las cosas en las cuales más creía entonces, las cosas en las cuales más creo
ahora, son los llamados \say{cuentos de hadas}.
Me parecen ser las cosas más razonables. No son fantasías; comparadas con ellos, otras cosas son
las fantásticas. Comparadas con ellos, la religión y el racionalismo son anormales, a pesar de que la
religión es anormalmente cierta y el racionalismo anormalmente equivocado. El país de las hadas, no es
más que la radiante patria del sentido común. No es la tierra la que juzga al cielo sino el cielo el que juzga
a la tierra; y del mismo modo, a lo menos para mí, no era la tierra la que criticaba al país de los elfos, sino
el país de los elfos el que criticaba a la tierra. Conocí el lenguaje de las habas antes de haberlas probado;
estaba seguro de que existía el \say{hombre de la luna}, antes de estar seguro de que la luna existía. Todo iba
de acuerdo con la tradición popular. Los poetas modernos de segunda categoría son naturalistas y hablan
de la enramada o del arroyo; pero los cantantes de la antigua épica fabulosa, eran supernaturalistas y
hablaban de los dioses de la enramada y del arroyo. Eso es lo que quieren significar los modernos, cuando
dicen que los antiguos no \say{apreciaban la Naturaleza}, porque, dicen ellos, la Naturaleza, es divina. Las
viejas ayas no hablan a los niños del pasto, sino do las hadas que bailan sobre el pasto; y los antiguos
griegos, no podían ver los árboles distraídos por las dríades.

Pero aquí me ocupo en demostrar que la ética y la filosofía vienen, alimentándose uno con cuentos
de hadas.

Si me ocupara de ellos detalladamente podría mencionar muchos nobles y saludables principios que
de ellos provienen. Allí está la caballeresca lección de \emph{Jack el caza gigantes}, según la cual se debe matar a los
gigantes porque son gigantescos. Es un motín valiente contra la soberbia. Porque el rebelde es más
antiguo que todos los reinos y el Jacobino tiene más tradición que el Jacobita.

Allí está la lección de \emph{Cenicienta} que es la misma lección que la del \emph{Magníficat}:  exaltavit humiles.\sidenote[][-5em]{Viene a significar algo así como \say{exaltar a los humildes}}
Allí, está la gran lección de \emph{La Bella y la Bestia}, según la cual una cosa debe ser amada, antes de
ser amable.

Allí está la terrible lección de \say{La Bella Durmiente}, que nos dice cómo la criatura humana al nacer
fue regalada con toda clase de bendiciones y no obstante, maldecida con la muerte; y cómo a veces la
muerte, puede dulcificarse hasta ser un sueño. Pero no me ocupo de los estatutos aislados del país de los
elfos, sino del espíritu de su ley en conjunto; su ley que aprendí antes de saber hablar y recordaré cuando
no pueda escribir.

Me ocupo, de una cierta manera de mirar la vida, creada en mí por los cuentos de hadas, pero que
desde entonces, fue humildemente confirmada por los hechos.

Podría exponerse de este modo: Existen ciertas continuidades o desenvolvimientos (cosas siguiendo
a otras cosas) que son razonables, en toda la extensión de la palabra. Que, en toda la extensión de la
palabra, son necesarias. Tales son las continuidades matemáticas y lógicas. Nosotros, en el país de las
hadas (que son las más razonables de todas las criaturas) admitimos esa razón y esa necesidad. Por
ejemplo, si las hermanas feas, son mayores que Cenicienta, es necesario que Cenicienta sea menor que las
hermanas feas. No hay otro camino. En torno a ese hecho Haeckel puede hablar todo lo que guste de
fatalismo. Si Juan es hijo de un molinero, un molinero es el padre de Juan. La fría razón lo decreta desde
su trono imponente: y nosotros, en el país de las hadas, nos sometemos. Si tres hermanos pasean a
caballo, allí andan complicados seis animales y dieciocho piernas: esto es verdadero racionalismo, v el
país de las hadas, rebosa de él. Pero cuando asomo la cabeza por encima del cerco de los elfos y
comienzo a estudiar el mundo natural, observo algo extraordinario. Observo que los hombres cultos y con
anteojos, hablaban de cosas actuales que sucedían, al amanecer, la muerte, etc.…, como si fueran
razonables o inevitables. Hablaban como si el hecho de que los árboles den frutas, fuera tan necesario
como el hecho de que dos árboles y un árbol son tres árboles. Pero no es tan necesario. Según la
experiencia del país de las hadas, que es la prueba de la imaginación, entre ambas cosas existe una
enorme diferencia. No es posible, imaginar que dos y uno, no sean tres. Pero fácilmente se imaginan
árboles que no dan fruta; o árboles que den candelabros dorados; o árboles de cuyas ramas cuelguen tigres
asidos por la cola.

Estos hombres con anteojos, hablaban de un tal señor Newton que fue golpeado por una manzana y
descubrió una ley. Pero esos hombres, no pueden llegar a ver la diferencia que existe entre una ley
necesaria, una ley razonable y el mero hecho de unas manzanas cayendo. Si la manzana golpeó la nariz a
Newton, la nariz de Newton golpeó la manzana. Esto es una necesidad cierta: porque no podemos
imaginar quo ocurra lo uno sin lo otro. Pero podemos concebir muy bien que la manzana no cayera sobre
su nariz; podemos imaginarla volando anhelosa por el aire para ir a golpear otra nariz cualquiera hacia la
cual sintiera una aversión más definida. En nuestros cuentos de hadas, siempre hemos conservado esta
diferencia penetrante entre la ciencia de las relaciones mentales en la cual existen leyes y la ciencia de los
hechos físicos en la cual no existen leyes sino solamente repeticiones extrañas. Creemos en milagros
corpóreos pero no en imposibilidades mentales. Creemos que un tallo de habas trepó hasta el cielo; pero
esto no altera nuestras convicciones en la cuestión filosófica de cuántas habas suman cinco.

Y aquí reside la perfección peculiar a la verdad y al tono de las fábulas infantiles. El hombre de
ciencia dice: \textquote{corte el cabo y la manzana caerá}; pero lo dice tranquilamente, como si una idea condujera
en realidad hacia la otra. La bruja en el cuento de hadas dice: \textquote{sopla el cuerno y caerá el castillo del
ogro}; pero no lo dice como si hubiera algo por lo cual evidentemente el efecto proviniera de la causa. Sin
duda, dio ese mismo consejo a muchos castillos, pero no pierde su aire expectante ni su razón. No hurga
en su cabeza hasta imaginar una conexión mental necesaria entre el cuerno y el castillo tambaleante. Pero
los científicos hurgan en sus cabezas hasta imaginar una conexión mental entre la manzana abandonando
el árbol y la manzana llegando al suelo. Hablan como si realmente hubieran descubierto no sólo una
cantidad de hechos maravillosos, sino una verdad que conecta entre sí esos hechos. Hablan como si la
conexión física de dos cosas extrañas las conectara también filosóficamente. Sienten que por el hecho de
que una cosa incomprensible constantemente siga a otra cosa incomprensible, de algún modo las dos
forman algo comprensible. Dos jeroglíficos negros formando una respuesta blanca.

En el país de las hadas evitamos usar la palabra \say{ley}; pero en el país de la ciencia, le son
particularmente afectos. De ahí que llamen \say{Ley de Grimm} a alguna conjetura interesante sobre cómo
los pueblos olvidados pronunciaban el alfabeto. Pero la ley de Grimm es mucho menos interesante que los
cuentos de hadas de Grimm.\nota{Los hermanos Grimm, Jacob (1785\enraya1863) y Wilhelm (1786\enraya1859), fueron académicos alemanes, filólogos e investigarodes culturales. Muchos de los cuentos populares europeos fueron recolectados por ellos, y su versión es a la que hace referencia Chesterton a lo largo de este capítulo.} Los cuentos, por lo menos, son verdaderamente cuentos, mientras que la ley,
no es una ley.

Una ley implica que conozcamos la naturaleza de su generalización y de su establecimiento, no que
tengamos sólo una vaga idea de sus efectos. Si existe una ley, según la cual los rateros deben ir a la
cárcel, implica que hay una conexión mental imaginable entre la idea de prisión y la idea de ratería y
sabemos cuál es la idea. Podemos explicar por qué privamos de Libertad a un hombre que se toma
libertades. Pero no podemos decir por qué un huevo pudo convertirse en pollo, del mismo modo que no
podemos decir por qué un oso pudo convertirse en príncipe. Como ideas, la de huevo y la de pollo, son
más remotas entre sí que la de oso y la de príncipe, porque en sí, no hay huevos con aspecto de pollo
mientras que hay príncipes con aspecto de oso.

Concedido que existen ciertas transformaciones, es esencial que las consideremos desde el punto de
vista filosófico de los cuentos de hadas y no a la antifilosófica manera de la ciencia y de las \say{Leyes de la
Naturaleza}. Cuando nos pregunten por qué los huevos se convierten en aves y por qué los frutos caen en
otoño, debemos contestar exactamente como la contestaría el hada madrina a Cenicienta, si ésta le
preguntara por qué los ratones se convertían en caballos y sus vestidos desaparecían al dar media noche.

Debemos contestar que es magia. No es una ley, porque no entendemos su fórmula general. No es
una necesidad, porque a pesar de dar prácticamente por descontado que esas cosas sucedan, no tenemos
derecho a decir que siempre han de suceder. El hecho de que contemos con el curso ordinario de los
acontecimientos, no es (según imaginó Huxley) argumento suficiente para fundar la inmutabilidad de una
ley. Y no contamos con el curso ordinario de las cosas, sino que apostamos sobre él. Nos arriesgamos a la
remota posibilidad de un milagro, como lo haríamos con un pastel envenenado o con un cometa
destructor del mundo. Lo damos por descontado, no porque es un milagro y por consecuencia una
excepción. Todos los términos empleados en los libros de ciencia, \say{ley}, \say{necesidad}, \say{orden},
\say{tendencia} y otros en ese estilo, son en realidad inintelectuales porque implican una síntesis intrínseca
que no poseemos.

Las únicas palabras que siempre me satisficieron para describir la Naturaleza, son las empleadas en
los libros de cuentos de hadas, tales como \say{encanto}, \say{hechizo}, \say{encantamiento}. Expresan la
arbitrariedad del hecho y de su misterio. Un árbol da frutas porque es un árbol mágico. El agua cae de la
montaña porque está embrujada.
El sol brilla porque está encantado.

Niego absolutamente que esto sea fantástico o aun místico. Más tarde podremos tener algún
misticismo; mas para hablar de las cosas, este lenguaje de cuentos de hadas es simplemente racional y
agnóstico. Emplearlo, es mi único camino para expresar con palabras mi clara y definida percepción, de
que una cosa es muy distinta a otra; que no existe conexión lógica entre volar y poner huevos. Místico es
el hombre que habla de \say{una ley} sin nunca haberla visto. Del mismo modo que es estrictamente
sentimental el corriente hombre de ciencia. Es un sentimental en este sentido; se deja empapar y arrastrar
por meras asociaciones. Ha visto pájaros volando y poniendo huevos con tanta frecuencia, que siente que
entre las dos ideas, debe existir alguna conexión tierna y soñadora, cuando en realidad no hay ninguna. El
amante abandonado puede ser incapaz de disasociar a la luna de su amor perdido; así como el materialista
es incapaz de disasociar a la luna de las mareas. En ambas cosas no existía más conexión que la de haber
sido vistas simultáneamente. Un sentimental puede llorar por el perfume de una flor de manzano, a causa
de que por una nebulosa asociación personal de ideas, le recuerda su infancia. Así el profesor materialista
(aunque esconda sus lágrimas) es un sentimental, porque por una nebulosa asociación personal, la flor de
manzano le recuerda las manzanas. Pero el frío racionalista del país ele las hadas, en lo abstracto no ve
por qué el manzano no ha de dar tulipanes rojos; en su patria a veces los da.

Sin embargo, este asombro no es una mera fantasía derivada de los cuentos de hadas; al contrario,
de él deriva todo el fuego de los cuentos de hadas. Así como a todos nos gustan los cuentos de amor,
porque hay en ellos un instinto de sexo, a todos nos gustan las fábulas asombrosas porque tocan la fibra
del antiguo instinto de asombro. Esto lo prueba el hecho de que cuando somos muy niños, no necesitamos
cuentos de hadas; solamente necesitamos cuentos; La vida resulta bastante interesante. Un chico de siete
años se entusiasma, si le dicen que Tomás abrió una puerta y vio un dragón. Pero un chico de tres años, se
entusiasmará si le dicen que Tomás abrió una puerta. A los chicos les gustan los cuentos románticos; pero
a los bebés les gustan los cuentos realistas, porque los encuentran románticos. En realidad, un bebé,
pienso que aproximadamente, es la única persona que puede leer una novela realista moderna, sin
aburrirse.

Esto prueba que aun las fábulas infantiles sólo son eco de un sobresalto, casi prenatal, de interés y
de asombro. Estas fábulas dicen que las manzanas son doradas, con el único fin de resucitar el momento
olvidado en que descubrimos que eran verdes. Dicen que corren ríos de vino, para recordarnos por un
loco momento, que corren ríos de agua. Dije que esto es completamente razonable y aún agnóstico. Y
ciertamente que sobre este punto, estoy con el agnosticismo; cuyo nombre mejor es Ignorancia.

Todos hemos leído en libros científicos y por cierto también en las novelas, la historia del hombre
que olvidó su nombre.

Ese hombre camina por las calles y puede verlo y apreciarlo todo; sólo no puede recordar quién es.

Bien, cada hombre, es ese hombre de la historia. Cada hombre ha olvidado quién es. Es terrible
comprender el cosmos pero nunca comprender el \say{ego}; el \say{yo}, es más remoto que cualquier estrella.

Amarás al Señor tu Dios, pero nunca lo comprenderás. Todos padecemos de la misma calamidad mental;
todos hemos olvidado nuestros nombres. Todos hemos olvidado lo que somos. Lo que llamamos sentido
común, y racionalidad y practicidad y positivismo, significa que por ciertas regulaciones de nuestra vida,
olvidamos que hemos olvidado. Todo lo que llamamos espíritu, y arte y éxtasis, significa que solamente
por un magnífico instante, recordamos que habíamos olvidado.

Pero a pesar de que (como el hombre sin memoria en la novela) caminamos por las calles con una
especie de admiración tardía, todavía es con admiración. Es admiración en inglés y no puramente
admiración en latín.

El asombro tiene un positivo elemento de alabanza. Este es el próximo mojón que hemos de pasar
para hallarnos definitivamente resueltos en nuestro camino a través del país de las hadas. En el próximo
capítulo hablaré del aspecto intelectual del optimismo y del pesimismo; tanto manto tengan uno. Aquí
sólo trato de describir las enormes emociones que no pueden ser descritas. Y la emoción más fuerte de la
vida, fue tan hermosa como desconcertante.

Fue un éxtasis porque fue una aventura; fue una aventura porque fue una oportunidad. La bondad de
los cuentos de hadas no se afectó porque en ellos puedan haber más dragones que princesas; ya era
bondad figurar en un cuento de hadas. La prueba de toda felicidad es la gratitud; y me siento agradecido,
pese a no saber a quién.

Los niños están agradecidos a Santa Claus, cuando llena sus medias de juguetes y dulces. ¿Podría
no estar agradecido a Santa Claus cuando ha llenado mis medias con dos piernas milagrosas?
Agradecemos a la gente regalos de cumpleaños como cigarros y zapatillas.

¿Puedo no agradecer a nadie el regalo de cumpleaños de mi nacimiento?
Luego, allí están esos dos sentimientos indefinibles e indiscutibles. El mundo era un choque; pero
no era puramente chocante; la existencia fue una sorpresa, pero fue una sorpresa agradable. De hecho, mis
primeras impresiones se manifestaron como un jeroglífico alojado en mi cabeza desde la infancia. La
pregunta era: \textquote{¿Qué dijo la primera rana?}; y la respuesta era: \textquote{¡Señor, cómo me haces saltar!} Esto
expresa brevemente todo lo que estoy diciendo. Dios hizo saltar a la primera rana; pero la rana prefiere
saltar. Mas cuando estas cosas se han puesto de acuerdo, comienza el segundo gran principio de la
filosofía feérica.\nota[][-4em]{Adjetivo que denota pertenencia o relatividad a las hadas}

Puede hallarlo quienquiera lea los \emph{Cuentos de Hadas} de Grimm o las delicadas
colecciones del señor Andrés Lang. Por darme el gusto de ser pedante, a ese principio le llamaré Doctrina
del goce condicional. Touchstone decía que el \say{si}, encerraba gran poder; conforme a la ética del país de
los elfos, todo poder reside en un \say{si}. El tono de las manifestaciones feéricas es siempre: \textquote{Usted podrá
vivir en un palacio de oro y zafiros si no pronuncia la palabra \say{vaca}}; o \textquote{Usted vivirá feliz con la hija del
Rey, si no le muestra un hongo}. La realización siempre está pendiente de una condición. Todas las cosas
estridentes y colosales concedidas, dependen de una pequeña cosa retenida. Todas las cosas terribles y
vertiginosas que se permiten, dependen de una cosa que se prohíbe. El señor W.\,B. Yeates,\nota[][-12em]{William Butler Yeats (1865\enraya1939) fue un poeta irlandes y una de las figuras más importantes de la literatura del siglo \textsc{xx}. Ganador del nobel de literatura de 1923, era nacionalista irlandes y odiaba el liberalismo y la democracia a partes iguales. Al final de su vida vio con entusiasmo a los fascismos europeos, llegando a componer marchas militares para los \say{Blueshirts} (movimiento paramilitar irlandes, de corte fascista).} en su
exquisita y penetrante poesía feérica, describe a los genios como alegales; en una inocente anarquía,
cabalgan sobre los caballos desenfrenados del aire: \textquote{Cabalgan en las crestas de las olas o sobre el
desorden de las mareas, y bailan sobre las montañas como llamaradas}.

Que el señor Yeates no comprende el país de las hadas, es penoso decirlo. Pero lo digo. Es un
irlandés irónico lleno de reacciones intelectuales. Pero no es bastante estúpido para comprender el país de
las hadas. Las hadas prefieren a la gente de yugo, como yo; gente que bosteza y tuerce la boca y hace lo
que se les dice. El señor Yeates, ve en el país de los elfos, toda la justa insurrección de su propia raza.

Pero la alegalidad de Irlanda, es una insurrección Cristiana, fundada en la razón y en la justicia;
pero el verdadero ciudadano del país de las hadas, se rebela obedeciendo a algo que no comprende en
absoluto. En los cuentos de hadas, la felicidad incomprensible depende de una incomprensible condición.

Se abre una caja y todos los demonios vuelan libertados. Se olvida una palabra y las ciudades perecen. Se
enciende una lámpara y el amor huye. Se recoge una flor y una vida termina.

Se come una manzana y se pierde la esperanza en Dios.

Este es el tono de los cuentos de hadas; y ciertamente no es un tono de insurrección ni de libertad, a
pesar de que, bajo una mezquina tiranía moderna, por comparación los hombres pueden pensar que eso es
libertad. Los que salen de la Cárcel de Portland, pueden creer que en Fleet Street\nota{Centro de la vida periodística.} se es libre; pero un
estudio del asunto hecho desde más cerca, probará que tanto las hadas como los periodistas son esclavos
del deber. Por lo menos las hadas madrinas son tan severas como otras madrinas. Cenicienta recibió un
coche traído del País de las Maravillas y un cochero traído de ninguna parte, pero también recibió orden
de volverse a las doce. Tenía un zapato de cristal; y no puede ser una coincidencia que el vidrio sea una
sustancia tan común entre la gente científica. Esta princesa vive en un palacio de cristal; aquella sobre una
colina de cristal; ésta vé todas las cosas en un espejo; todas pueden vivir en casas de vidrio mientras no
tiren piedras. Porque este cristal delgado y reluciente, en todas partes es símbolo de un hecho: que la
felicidad es reluciente pero frágil, como la sustancia que más fácilmente destruye una mucama o un gato.

Y este sentimiento de los cuentos de hadas, arraigó en mí y llegó a ser también mi sentimiento hacia todo
el mundo. Sentí y siento que en sí la vida es tan brillante como un brillante y tan frágil como un vidrio de
ventana; y cuando se enfrentó a los cielos con el cristal terrible, recuerdo que me estremecí. Tenía miedo
de que Dios dejara de sostener al mundo y el mundo cayera estruendosamente.

Recuérdese no obstante, que ser rompible, no es lo mismo que ser perecedero. Golpee un vidrio y
no durará un instante; no lo golpee y durará cien años. Tal parece haber sido la alegría del hombre en el
cielo y en la tierra; la felicidad dependía de abstenerse de hacer algo que en cualquier momento podría
hacerse y que con frecuencia no era evidente la razón por la cual no debía ser hecho. Aquí el punto es que
a mí eso no me parece injusto. Si el tercer hijo del molinero dijera al hada: \textquote{Explícame por qué en el
palacio de las hadas no me puedo parar sobre la cabeza}; la otra, sinceramente pudo responder: \textquote{Bien; si
en eso estamos, explícame el porqué del palacio de las hadas.} Si Cenicienta dice: \textquote{¿Por qué tengo que
dejar el baile a las doce?}. Su madrina podría contestarle: \textquote{¿Por qué es que puedes estar allí hasta las
doce?} Si en mi testamento le dejo a un hombre diez elefantes que hablan y cien caballos alados, no
puede quejarse, porque las condiciones compensan la ligera excentricidad del regalo. A caballo alado no
se le miran los dientes.

Y me parece que la existencia, en sí, era una regalo excéntrico como ese y que no podía quejarme
de no entender las limitaciones de mi visión, cuando no entendía la visión que limitaban. El marco, no era
más extraño que la pintura. La condición muy bien podría ser tan desorbitada como la visión; podría ser
tan asombrosa como el sol, tan escurridiza como el agua, tan fantástica y terrible como los árboles
gigantescos.

Por esta razón (que podríamos llamar filosofía del hada madrina) nunca pude adherirme a los
jóvenes de mis tiempos, para sentir, lo que ellos llamaban \say{sentimiento general de rebelión}. Me habría
opuesto (esperemos) a toda regla perniciosa; pero de éstos y sus definiciones me ocuparé en otro capítulo.

Lo cierto es que no me sentía dispuesto a sostener cualquier regla, por el sólo hecho de ser misteriosa. A
veces, se reprimieron los estados con procedimientos estúpidos; romper bastones, o pagar un grano de
pimienta.

Yo quería reprimir al inmenso estado del cielo y de la tierra, con alguna de esas fantasías feudales.

Nunca podría ser más loca que el hecho de que me fuera permitido hacerlo. En este peldaño, sólo
doy un ejemplo ético para explicar lo que quiero decir. Nunca me pude mezclar con la generación
incipiente en el murmullo común contra la monogamia; porque ninguna restricción al sexo me parecía tan
extraña e inesperada como el sexo mismo. Tener la posibilidad, como Endimión, de enamorar a la luna y
luego quejarse porque Júpiter guardaba sus lunas propias en un harem (alimentado de cuentos de hadas
como el de Endimión), me parecía todo ello un anticlímax. Conservarse para una mujer, es poco precio
para lo mucho que es ver una mujer. Quejarme porque me casé solamente una vez, es como quejarme
porque he nacido una vez sola. Sería desproporcionada esa queja, frente a la terrible conmoción de que se
está hablando. Oponerse a la monogamia evidenciaba no una exagerada sensibilidad de sexo, sino una
curiosa insensibilidad a él. Es un tonto el hombre que se queje porque no puede entrar al Paraíso por
cinco puertas al mismo tiempo. La poligamia es una falta en la realización del sexo; es como el hombre
que pela cinco peras sencillamente porque está distraído. En su elogio a las cosas amables, los estetas
llegaron al último límite de la locura del lenguaje. Lloran por los cardos y caen de rodillas ante un
escarabajo.

No obstante, su emotividad, nunca, ni por un instante llegó a conmoverme; por esta razón: nunca se
les ha ocurrido pagar su placer ni con un sacrificio simbólico.

Los hombres (lo he sentido), son capaces de vivir apurados cuarenta días, con tal de oír cantar a un
mirlo. Los hombres pueden pasar por el fuego para encontrar una hierba extraña. Sin embargo, estos
amantes de la belleza no podrían mantenerse sobrios por eI mirlo. No pasarían por el común matrimonio
cristiano en agradecimiento a la hierba. Con la moral corriente seguramente se podrían pagar los goces
extraordinarios. Oscar Wilde dijo que las puestas de sol no tienen valor porque no podemos pagarlas.

Pero Oscar Wilde se equivocaba. Podemos pagar las puestas de sol, con sólo no ser Oscar Wilde.

Bien; dejé los cuentos de hadas por el suelo de la nursery; y desde entonces no encontré libros más
sensatos.

Dejé a la niñera, guardiana de la tradición y la democracia; y no he encontrado otro tipo moderno
tan radicalmente sano, tan sanamente conservador. Pero el asunto del comentario importante y central,
está aquí: cuando por primera vez fui al mundo moderno, hallé que el mundo moderno, en dos puntos, se
encontraba decididamente opuesto a mi niñera y a los cuentos infantiles. Tardé mucho tiempo para
descubrir que el mundo moderno se equivocaba y mi niñera no. Lo realmente curioso era esto: que el
pensamiento moderno contradecía esas creencias fundamentales de mi infancia, en sus doctrinas más
esenciales. He explicado que los cuentos de hadas me infundieron dos convicciones: primera, que este
mundo es un lugar terrible y sorprendente, que podía haber sido distinto y es muy agradable; segunda, que
ante este salvajismo, y encanto, muy bien se puede ser modesto y someterse a las más extrañas
limitaciones de tan extraña bondad. Pero encontré a todo el mundo moderno corriendo como una
marejada contra mis dos ternuras, y el colapso del encontrón, creó dos sentimientos repentinos y
espontáneos, que conservé desde entonces y han adquirido ya, solidez de convicciones.

Primero encontré al mundo moderno hablando de fatalismo científico; decían que cada cosa es
como hubo de haber sido siempre, por ser conformada sin error, desde el principio. La hoja del árbol es
verde porque nunca pudo ser de otro color. El filósofo de los cuentos de hadas, se alegra de que la hoja
sea verde, porque pudo haber sido colorada. Siente como si se hubiera vuelto verde un instante antes de
mirarla. Está satisfecho de que la nieve sea blanca, en el sentido estrictamente razonable de que pudo
haber sido negra. Cada color tiene en sí una cualidad inconfundible; cómo si fuera elegida.

El rojo de las rosas de jardín, no es sólo decisivo sino dramática, como repentinas salpicaduras de
sangre. El filósofo de los cuentos de hadas, siente como si algo se hubiera hecho. Pero los grandes
deterministas del siglo XIX, se opusieron vigorosamente a esta sensación natural de que algo ha sucedido
un momento antes.

Según ellos, desde el principio del mundo, nunca en realidad ha sucedido nada. Nunca había
sucedido nada desde el suceso de la existencia: y ni están muy seguros de la fecha en que sucedió.

El mundo moderno tal como lo encontré, se afirmaba en el Calvinismo moderno, por la necesidad
de que las cosas sean lo que fueron. Pero cuando comencé a pedir pruebas de esta inevitable repetición
descubrí que realmente no las tenían, a no ser el hecho de que las cosas se repetían. Mas la mera
repetición, me presentaba todo en una forma bastante más extraña que racional. Era como si hubiera visto
en la calle una nariz extraña, la olvidara por considerarla accidental, y luego viera seis narices más con la
misma estructura asombrosa.

Por un momento, debí haberme imaginado que se trataba de alguna sociedad secreta local. Así, un
elefante con trompa es extraño; pero todos los elefantes con trompa puede parecer una especie de
complot. Aquí hablo solamente de una emoción, y de una emoción obstinada y al mismo tiempo sutil.

Pero la repetición en la naturaleza, a veces parecía ser una repetición enervada, como la del maestro de
escuela enfurecido, que repite la misma cosa una y otra vez. El pasto parecía señalarme con todos los
dedos a un tiempo; la multitud de estrellas parecían inclinadas buscando comprensión. El sol se me
mostraría siempre, aunque salga mil veces. La repetición del universo llegó a adquirir el ritmo
enloquecido de un encantamiento; y comencé a vislumbrar una idea.

Todo el imponente materialismo que domina a las mentes modernas, descansa ulteriormente en una
presunción; en una presunción falsa, Se supone que es muerta una. cosa que constantemente se repite;
algo como un engranaje relojero. La gente siente que si el mundo fuera personal variaría; si el sol tuviera
vida, bailaría.

Esto es un sofisma, aún si se le relaciona con hechos conocidos. Porque en los asuntos humanos, la
variación generalmente la introduce la muerte y no la vida; el decaimiento o el quebranto de la fuerza o el
deseo.

Un hombre varía sus movimientos por un leve elemento de fracaso o de fatiga. Se sube a un
ómnibus porque está cansado de caminar o camina porque está cansado de estarse quieto. Pero si su vida
y su alegría fueran tan gigantescas como para no cansarse nunca de ir a Islington, podría ir a Islington tan
regular y continuadamente como el Támesis va a Sheerness. Y la misma velocidad y el éxtasis propios de
su vida, llegarían a la quietud de la muerte.

El sol se levanta cada mañana; yo no me levanto cada mañana, pero lo que me diferencia de él no es
mi actividad sino mi inacción. Y para exponer el punto en una frase popular, podría decir que el sol se
levanta regularmente porque nunca se cansa de levantarse. Podría observarse lo que quiero decir, por
ejemplo en los niños, cuando descubren un juego o una broma que les proporciona especial alegría. Un
niño se golpea rítmicamente los talones, a causa de un desborde y no de una carencia de vida. Porque los
niños rebosan vitalidad por ser en espíritu libres y altivos; de ahí que quieran las cosas repetidas y sin
cambios. Siempre dicen \textquote{hazlo otra vez}; y el grande vuelve a hacerlo aproximadamente hasta que se
siente morir. Porque la gente grande no es suficientemente fuerte para regocijarse en la monotonía. Pero
tal vez Dios sea bastante fuerte para regocijarse en ella. Es posible que Dios diga al sol cada mañana:
\textquote{hazlo otra vez}, y cada noche diga a la luna: \textquote{hazlo otra vez}.

Puede que todas las margaritas sean iguales, no por una necesidad automática; puede que Dios haga
separadamente cada margarita y que nunca se haya cansado de hacerlas iguales. Puede que Él, tenga el
eterno instinto de la infancia; porque pecamos y envejecimos, y nuestro Padre es más joven que nosotros.

La repetición en la Naturaleza puede no ser un mero recomenzar; puede ser un teatral \say{todavía}. El Cielo
puede decir \say{todavía}, al pájaro que puso un huevo.

Si el ser humano concibe y trae al mundo un niño humano, y no un pez, ni un murciélago, ni una
quimera, la razón no puede ser que estemos encaminados a un destino animal, sin vida y sin motivo.

Puede ser que nuestra pequeña tragedia haya conmovido e interesado a los dioses que la admiren desde
sus galerías estrelladas, y que al fin de cada drama humano, el hombre sea llamado una y otra vez a
escena.

La repetición puede continuar por millones de años y en cualquier momento puede detenerse. El
hombre puede permanecer sobre la tierra generaciones tras generaciones y sin embargo, cada nacimiento,
positivamente, puede ser su última aparición sobre la tierra. Esta fue mi primera convicción, creada por la
conmoción de mis emociones infantiles, al encontrarse, a medio camino, con las creencias modernas.

Siempre había intuido vagamente que los hechos eran milagrosos, en el sentido de que son sorprendentes:
ahora empiezo a creerlos milagrosos en el sentido más estricto, de que son premeditados. Quiero decir
que son, o pueden ser, ejercicios repetidos de una voluntad. En realidad, siempre creí que el mundo
implicaba magia: luego pensé que quizá implicara un mago. Y esto aguzaba una emoción profunda,
siempre presente y subconsciente: que este mundo nuestro tiene un motivo; y si hay un motivo hay una
persona. Siempre sentí que la vida, era en primer lugar como una historia; y si hay una historia, hay un
relator.

Pero el pensamiento moderno también golpeó a mi segunda tradición humana. Iba contra mi feérico
sentimiento respecto a las condiciones y limitaciones estrictas. Al pensamiento moderno, le gustaba
hablar de expansión y amplitud. Herbert Spenser se habría disgustado mucho si alguien le hubiera
llamado imperialista, de ahí que es profundamente lamentable que nadie lo haya hecho. Pero era un
imperialista y del tipo más bajo. Popularizó la despreciable idea de que la magnitud del sistema solar,
debía sobreponerse al dogma espiritual del hombre. ¿Por qué un hombre habría de entregar su dignidad al
sistema solar más bien que a una ballena? Si es simplemente el tamaño lo que prueba que el hombre no es
imagen de Dios, entonces, la ballena puede ser la imagen de Dios; una imagen un tanto deforme: lo que se
podría llamar un retrato de la escuela impresionista. Es completamente inútil argüir que, comparado con
el cosmos, el hombre es pequeño; porque comparado con el árbol más próximo, el hombre siempre fue
pequeño. Pero Herbert Spenser, en su aturdido imperialismo, insistirá en que, de cierta manera, hemos
sido conquistados y anexados al universo astronómico. Hablaba de los hombres y de sus ideales, como
hablaría de Irlanda y sus Ideales, el Unionista\nota[][-2em]{Partidario de la anexión de Irlanda a Gran Bretaña.} más insolente. Convirtió a la humanidad en una
nacionalidad pequeña. Y su mala influencia se advierte aún en los más vigorosos y honorables autores
científicos: notablemente en las primeras obras del señor H.\,G. Wells. En forma exagerada, muchos
moralistas han presentado como perversa a la tierra. Pero el señor Wells, y sus seguidores, presentaron al
Cielo como más perverso aún. Levantaríamos la mirada a las estrellas, desde donde nos viniera nuestra
ruina…
Pero la expansión de la cual hablo, era mucho más perversa. He observado que el materialista,
como el loco, está en prisión; en la prisión de un pensamiento. Y esa gente parece hallarla especialmente
inspiradora, ya que insiste en que la prisión es muy amplia. La amplitud de ese mundo científico, no nos
ofrece ninguna novedad, ningún alivio. Su cosmos siempre seguía su marcha, pero ni en la constelación
más extraña había nada realmente interesante; nada como, por ejemplo, de perdón; nada de libre albedrío.

La grandeza o la infinitud del cosmos materialista, no le agregaba nada. Era como decir al
prisionero de la cárcel de Reading, que se alegrara de oír que la cárcel ahora alcanza a cubrir medio
condado. El guardián no tendría nada que mostrarle al hombre, excepto más y más largos corredores de
piedra, tétricamente alumbrados, y vacíos de todo lo que es humano. Así, esos expansores del universo,
no tenían nada que mostrarnos, excepto más y más corredores del espacio, alumbrados por lúgubres soles
y vacíos de todo lo que es divino.

En el país de las hadas existía una verdadera ley; una ley no podía ser quebrantada, porque la
definición de \say{ley}, se refiere a algo que puede quebrarse. Pero la maquinaría de esta prisión cósmica, era
algo inquebrantable; porque nosotros mismos, éramos sólo una parte de la maquinaria. O éramos
incapaces de hacer algo, o estábamos destinados forzosamente a hacerlo. La idea mística de lo
condicional desaparecía completamente; no era posible tener la firmeza de observar la ley ni el placer de
quebrantarla. La amplitud de este universo, no tiene nada de la rebelión fresca y aireada que hemos
alabado en el universo del poeta. Este universo moderno, es literalmente un imperio; es decir, es vasto
pero no libre. Se pueden recorrer muchas habitaciones, a cual más grande, pero sin ventanas; habitaciones
grandiosas con perspectivas babilónicas; pero no es posible descubrir nunca, ni la ventana más pequeña ni
el susurro del aire libre que se quedó afuera.

Sus paralelos infernales parecían prolongarse más allá de la distancia; mas para mí, todas las cosas
buenas deben llegar a un punto, por ejemplo, las espadas. Encontrando los alardes del cosmos, tan poco
satisfactorios para mis emociones, comencé a profundizar el asunto; y pronto hallé que todos sus
desplantes eran aún más infundados de lo que podía preverse. Según los materialistas el cosmos era una
cosa, puesto que era regido por una ley inquebrantable. Sólo que (dicen ellos) como es una cosa, es
también la única cosa que existe. ¿Por qué entonces preocuparse de llamarla amplia? Sería igualmente
sensato, llamarla pequeña.

Un hombre podía decir: \textquote{me gusta este cosmos vasto, con su multitud de estrellas y su
muchedumbre variada de criaturas.} Pero si es por eso, ¿por qué no diría el hombre: \textquote{me gusta este
cosmos pequeño e íntimo, con su discreto número de estrellas y su provisión de vida tan breve, como a mí
me gusta}? Una cosa es tan sensata como la otra; ambos, son puramente sentimientos. Es simplemente un
sentimiento regocijarse porque el sol es más vasto que la tierra; es un sentimiento tan sensato como el de
regocijarse porque el sol no sea más vasto que ella.

Un hombre se inclina a sentir una determinada emoción frente a la amplitud del mundo. ¿Por qué no
podría estar inclinado a sentirla frente a su pequeñez? Y casualmente ocurre que yo he sentido esa última
emoción. Cuando se quiere a algo, uno llama a ese algo con diminutivos, aunque se trate de un elefante o
de un guarda espalda gigantesco. La razón es que, cualquier cosa que uno imagine o conciba completa,
por inmensa que sea, puede concebirse como si fuera pequeña. Si el bigote militar no se asociara con una
espada, o los colmillos del elefante no sugirieran una cola, el bigote y los colmillos, serían vastísimos,
porque serían inconmensurables. Desde el momento en que es posible imaginar un guardaespalda, es
posible imaginar un guardaespalda pequeño. Y desde el momento en que es posible ver realmente un
elefante, es posible comenzar a llamarlo \say{Tiny}. Si usted puede hacer una estatua de algo, de ese, algo
igualmente puede hacer una estatuita.

Esa gente materialista proclama que el universo es algo coherente; pero no les gusta el universo.

Pero a mí el universo me gustaba terriblemente y quería dirigirme a él con un diminutivo. Con frecuencia
lo hice; y nunca pareció ofenderse. Luego, y sinceramente sentí que esos oscuros dogmas de la vitalidad,
se expresaban mejor llamando \say{pequeño}, al mundo, y no llamándole vasto. Porque había una especie de
displicencia hacia lo infinito que era el reverso de la orgullosa y piadosa consideración que yo sentía por
el valor inmenso y el peligro de la vida. Los materialistas se mostraban con ella de una lúgubre
prodigalidad; yo la sentí como una especie de ahorro sagrado. Porque la economía es mucho más
romántica que el despilfarro. Para ellos, las estrellas eran una entrada sin fin de medios centavos; pero yo,
por el sol dorado y la luna de plata, me sentí como se siente un escolar que tiene en su haber una esterlina
de oro, y un peso plateado. Estas convicciones subconscientes se manifiestan mejor con el colorido y el
tono de ciertos cuentos. Por eso dije que solamente las historias de magia pueden expresar mi sensación
de que la vida no es sólo un placer sino también una especie de privilegio excéntrico. Puedo expresar esa
otra sensación de la confortable intimidad del cosmos, refiriéndome a otro libro siempre leído en la
infancia \emph{Robinson Crusoe}, que he leído más o menos recientemente y que debe su eterna frescura al
hecho de que celebra la poesía de las limitaciones, y por consiguiente, hasta al silvestre romanticismo de
la prudencia. Crusoe es un hombre, recién evadido del mar que se ha instalado sobre un peñasco con unas
pocas comodidades. Lo más lindo del libro es la ennumeración de las cosas salvadas del naufragio. El
más grande de los poemas es un inventario. Cada utensilio de cocina se convierte en el utensilio ideal,
porque Crusoe pudo haberlo dejado caer al mar. Es un buen ejercicio para las horas ingratas o vacías del
día, mirarlo todo y pensar cuán feliz uno puede sentirse de haberlo salvado del barco zozobrante y llevado
luego a la isla solitaria.

Y es mejor aún el ejercicio de recordar cómo todo se salvó por un pelo: cada cosa que tenemos se
salvó de un naufragio. Cada hombre ha tenido una horrible aventura: como un oculto nacimiento fuera del
tiempo; él, no era; igual que los niños que nunca llegan a la luz. En mi infancia se hablaba mucho de
hombres de genio disminuidos o arruinados; y era común decir de muchos de ellos que eran: \textquote{Grandes
Pudieron Ser.} Para mí es un hecho más cierto y sorprendente que cualquier hombre que cruzó por la
calle es un: \say{Grande Pudo No Haber Sido.}
Pero aunque parezca tonto, realmente sentí como si el orden y el número de cosas, fueran los
románticos restos del barco de Crusoe. Que haya dos sexos y un sol, era como que hubieran allí dos armas
de fuego y un hacha. Era absolutamente urgente que ninguna de esas cosas se perdiera; pero en cierta
forma era bastante extraño, que a esas, no se pudiera agregar ninguna. Los árboles y los planetas parecían
cosas salvadas del naufragio; y cuando vi al Matterhorn\nota{Montaña famosa de Suiza; en francés Mont Cervin. Entre Nolais y el Piamonte.} me alegré de que no hubiera sido olvidado en la
confusión del momento. Me sentí económico con las estrellas, como si fueran zafiros (y así las llama
Milton en el Paraíso) ; me sentí avaro con las montañas. Porque el universo es todo, una sola joya y si es
natural en sentido figurado, decir inapreciable e incomparable a una joya, decirlo de esta joya sería
literalmente exacto. Este cosmos ciertamente es sin par y sin precio: porque no existe otro. Así concluye
con una imperfección inevitable este intento de decir lo indecible. Esta es mi ulterior posición frente a la
vida; los zurces para la simiente de la doctrina; lo que pensé en cierta forma obscura antes de poder
escribir, lo que sentí antes de poder pensar. Las resumo ahora para luego proseguir más fácilmente.

Sentí en mis huesos, primero, que este mundo no se explica a sí mismo. Puede ser un milagro con
una explicación sobrenatural; puede ser el truco de un hechizo con una explicación natural. Pero la
explicación del conjuro, si ha de satisfacerme, tiene que ser mejor que las explicaciones naturales que ya
he oído. Falsa o cierta, la cosa es de magia. Segundo, llegué a sentir que la magia tenía un significado, y
un significado debe tener alguien que lo signifique. En el mundo, había algo personal, como en una obra
de arte. Lo que significara aquello, lo significaba violentamente. Tercero, hallé hermoso su objeto y sus
designios, pese a tener defectos, como serían por ejemplo los dragones.

Cuarto, comprendí que la forma adecuada de agradecerlo, es tener una especie de humildad y de
restricción: debemos agradecer a Dios la cerveza y el. Borgoña, no bebiéndolos en exceso. Debemos
también obediencia, a quienquiera nos haya hecho. Y finalmente, y lo más extraño, vino a mi mente una
vaga y vasta impresión de que en cierto modo, todo bien era un remanente a almacenar y a conservar
como sagrado; un remanente salvado de la primera ruina. El hombre ha salvado su bien como Crusoe
salvó sus bienes: los ha salvado de un naufragio. Todo eso sentí, y los años me dieron valor para sentirlo.

Yen todo ese tiempo, no había ni siquiera pensado en la teología Cristiana.
\finalCapituloOrnamento
