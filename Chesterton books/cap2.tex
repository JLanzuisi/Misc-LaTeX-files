\chapter{El maniático}%
\label{cha:El maniático}

\textsc{ni siquiera la gente mundana} comprende al mundo; confía enteramente en unas cuantas máximas
cínicas, que no son ni verdaderas.

Recuerdo una vez: caminaba con un próspero editor que me hizo una observación oída con
frecuencia; es casi un estribillo del mundo moderno. No obstante haberla oído con demasiada frecuencia,
o tal vez por esa misma razón, recién entonces, repentinamente, vi que tal observación no entrañaba
verdad alguna. El editor dijo de alguien: \textquote{ese hombre va a llegar; se tiene fe}.

Y recuerdo que mientras levantaba la cabeza para escuchar mejor, mi mirada cayó en un ómnibus
que llevaba escrito su punto de destino: \say{Hanwell}\,\nota{Nombre de un sanatorio de enfermos mentales, que se menciona con frecuencia en el libro.} y le contesté: 

---Quiere que le diga dónde están los
hombres que se tienen fe?, porque puedo decírselo. Conozco hombres que creen en sí mismos más
colosalmente que Napoleón y César. Puedo llevarlo hasta los tronos de los superhombres. Los que
realmente se tienen fe, están en un asilo de lunáticos.

Me respondió que no obstante esa creencia mía, había muchos hombres que se tenían fe y no
estaban en manicomios.

	---Sí; los hay ---repuse---, y usted más que nadie debe conocerlos. Aquel poeta borracho a quien usted
rechazó una tragedia lúgubre creía en sí mismo. Aquel viejo pastor que escribió una obra épica y de quien
usted se escondía en la trastienda, creía en sí mismo. Si usted consultara su experiencia de editor en vez
de consultar su horrenda filosofía individualista, sabría que haberse tenido fe, es una de las características
más comunes de los fracasados. Los actores que no pueden actuar, creen en sí mismos, y creen en sí
mismos los deudores que no le pueden pagar. Sería más cierto decir que un hombre fracasará porque se
tiene fe.

	>>Tener completa fe en sí mismo, no es exclusivamente un pecado. Tenerse fe absoluta es una
debilidad. Tenerse fe completa, creer completamente en sí mismo, es tener una creencia histérica y
supersticiosa. El hombre que la tiene, lleva la palabra \say{Hanwell} escrita en su frente, con tanta claridad
como la lleva escrita ese ómnibus.

Mi amigo el editor, dio esta profunda y efectiva réplica a mis conclusiones: \textquote{Y si un hombre no debe creer en sí mismo ¿en qué debe creer?}, Luego de una larga pausa contesté, \textquote{Iré a casa y escribiré un libro contestando a esa pregunta.}

Y este es el libro que escribí para contestarla.
Pero creo, que muy bien puedo empezarlo donde se inició nuestra discusión; en la vecindad de un
manicomio.

Los modernos maestros de la ciencia insisten, sobre la necesidad de basar toda investigación, en un
hecho. Los antiguos maestros de religión, se mostraron igualmente entusiastas de esa teoría. Empezaron
basándose en el hecho del pecado; un hecho tan evidente como las papas. Fuera posible o no fuera posible
que el hombre se purificara con ciertas aguas milagrosas, no cabe duda de que necesitaba purificación.

Pero algunos caudillos religiosos de Londres, relativamente materialistas, comenzaron en nuestros días a
negar, no la discutible milagrosidad del agua, sino a negar la indiscutible existencia de la mancha. Ciertos
teólogos modernos, discuten el pecado original, que es el único punto de la teología de la cristiandad que
puede ser realmente probado. Algunos discípulos del Reverendo R.\,J. Campbell, admiten la inocencia
divina que no pueden vislumbrar ni en sueños, pero niegan, especialmente, la culpa humana que pueden
ver hasta en la calle. Los santos más intransigentes y los más obcecados escépticos, por igual unos y
otros, tomaron el positivo mal, como punto de partida de sus argumentaciones.

Si es cierto (como evidentemente lo es) que un hombre puede hallar exquisito placer desollando un
gato, el filósofo religioso puede llegar a una de dos conclusiones. Debe, o negar la existencia de Dios, que
es lo que hacen los ateos; o bien negar la inalterable unión entre Dios y el hombre, que es lo que hacen los
cristianos. Parece que los nuevos teólogos piensan llegar a una solución altamente racionalista negando el
gato.

En esta situación especialísima, evidentemente ahora no es posible (con una esperanza remota de
aceptación general) comenzar como comenzaron nuestros padres, basándose en el hecho del pecado. Este
mismo hecho que fue para ellos (y es para mí) tan evidente como la luz, es precisamente el hecho que ha
sido discutido o negado. Pero aunque los modernos nieguen la existencia del pecado, supongo que no han
negado aun la existencia del manicomio.

Todavía estamos de acuerdo en que actualmente se produce un colapso intelectual tan innegable e
inconfundible como el derrumbe de una casa. Los hombres niegan el infierno; pero aun no niegan el
manicomio. Para no perder de vista los fines de nuestro primer argumento, el uno, el infierno, podría muy
bien reemplazar al otro, el manicomio. Quiero decir que, si una vez todos los pensamientos y las teorías
fueron juzgadas según condujeran al hombre a perder su alma, así, por nuestro presente punto de vista,
todas las teorías modernas pueden ser juzgadas, según conduzcan al hombre a perder sus cabales.

Es cierto que algunos hablan de la locura, con soltura y simpatía, como si se tratara de algo amable
y atrayente.

Pero un minuto de reflexión basta para demostrarnos que si hallamos belleza en la enfermedad,
generalmente es en la enfermedad de otro.

Un ciego puede ser pintoresco; pero se necesitan dos ojos para verlo pintoresco, y similarmente,
aun la más salvaje poesía de la locura, sólo puede percibirla el cuerdo. Para el insano su locura es
perfectamente prosaica porque es perfectamente cierta. El hombre que se cree pollo, siente en sí, toda la
insignificancia del pollo. Solamente porque vemos lo grotesco de su idea, podemos en encontrarla hasta
divertida; y solamente porque él no ve lo grotesco de su idea, lo han llevado a \say{Hanwell}. Abreviando, las
rarezas sólo sorprenden a la gente normal. Las rarezas no sorprenden a la gente rara. Por esa razón, la
gente normal se sabe divertir y la gente rara, siempre se lamenta del aburrimiento de la vida. Por esa
razón, las novelas modernas fenecen; y por esa razón, los cuentos de hadas permanecen. Los viejos
cuentos de hadas presentan al héroe como un joven humano completamente normal; sus aventuras son las
sorprendentes; y lo soprenden porque es normal. Pero en la novela psicológica moderna, el héroe es un
anormal; él, que es el centro, no es bien centrado. De ahí que las aventuras más extrañas no logren
sorprenderlo adecuadamente y que el libro resulte monótono. Se puede escribir la historia de un héroe
entre dragones; pero no la de un dragón entre dragones. El cuento de hadas relata lo que hará un hombre
cuerdo en un mundo loco. La novela, sobriamente realista de hoy, relata lo que un hombre esencialmente
loco, puede hacer en un mundo cuerdo.

Empecemos pues en el manicomio; desde este fatídico y fantástico albergue, iniciemos nuestro viaje
intelectual.

Ahora, si es que vamos a contemplar la filosofía de la cordura lo primero que hemos de hacer, es
destruir un grande y difundido error. Por todas partes se ha difundido la idea de que la imaginación,
especialmente la imaginación mística, es peligrosa para el equilibrio mental del hombre. En general se
tiene a los poetas como inseguros, desde el punto de vista psicológico; y generalmente se hace asociación
de ideas entre los laureles entrelazados y las pajas pinchadas en el pelo... Los hechos y la historia
desmienten tal interpretación. Muchos de los poetas, de los verdaderamente grandes poetas, han sido no
sólo perfectamente cuerdos sino extremadamente aptos para el comercio; y si Shakespeare alguna vez
contuvo caballos, fue porque era el hombre más indicado para contenerlos.

La imaginación no provoca la locura. Para ser exacto, lo que fomenta la locura es la razón. Los
poetas no enloquecen; los jugadores de ajedrez sí. Los matemáticos y los cajeros, se vuelven locos; pero
rara vez enloquecen los artistas que crean. Como podrá verse, en ninguna forma ataco la lógica: digo
solamente que el peligro de la locura reside en la lógica; no en la imaginación. La paternidad artística es
tan saludable como la física. Sin embargo, vale la pena destacar que cuando un poeta fue realmente
mórbido, comúnmente lo fue porque existía algún punto débil de racionalismo en su cabeza. Poe, por ejemplo, fue
realmente morboso; no porque fue poético, sino porque fue esencialmente analítico. Aun el ajedrez era
demasiado poético para él; le desagradaba porque había demasiados caballeros y castillos, como en un
poema.

Abiertamente manifestó su preferencia por las fichas negras, que sobre el damero parecían el
punteado de un diagrama. Quizás el ejemplo más contundente es este: que sólo un gran poeta inglés se
volvió loco.

Cowper.\nota[][-7em]{William Cowper (1731\enraya1800) fue un poeta británico. En 1973 fue enviado a un asilo para recuperarse, pues habia intentado suicidarse despues de un periodo de ansiedad y depresión. A partir de entonces Cowper estaría siempre al borde la de locura, llegando a convencerse de que estaba destinado a la perdición eterna. Sus poemas, que escribió siempre que pasaba por un momento oscuro en su vida, podrían haber sido en efecto la razón de que conservase algo de cordura.} Y decididamente, fue llevado a la locura por la ló\-gi\-ca; por la extraña lógica de la
predestinación. La poesía no fue su enfermedad sino su remedio; la poesía, en parte conservó su salud.

Por algunos momentos, pudo olvidar el rojo y sediento infierno al que lo empujaba su horrendo
necesitarismo, entre las extendidas aguas y los lirios blancos del Duse. Cowper, fue condenado por 
John Calvin y casi fue salvado por John Gilpin\nota[][9em]{Gilpin es el personaje principal de \say{The Diverting History of John Gilpin}, historia cómica escrita por Cowper en verso.}.

En todas partes, vemos que el hombre no enloquece por soñar. Los críticos son mucho más locos
que los poetas. Homero, es bastante tranquilo y completo; son sus críticos que lo destrozan en jirones de
extravagancia. Shakespeare, fue perfectamente él mismo; sólo algunos de sus críticos descubren que
Shakespeare fue otro. Y San Juan Evangelista, no obstante haber visto en su visión muchos monstruos
extraños, no vio criatura alguna tan salvaje como uno de sus comentaristas. El hecho general es claro. La
poesía es cuerda, porque flota sin esfuerzo en un mar infinito; la razón pretende cruzar el mar infinito y
hacerlo así finito.

El resultado es la exterminación mental; como lo fue la extenuación física para el señor Holbein.
Aceptarlo todo, es un ejercicio; entenderlo todo, es un esfuerzo. Lo único que desea el poeta, es
exaltación y expansión, un mundo para explayarse.
El poeta sólo pretende entrar su cabeza en el cielo.
El lógico es el que pretende hacer entrar el cielo en su cabeza. Y es su cabeza la que revienta.

Es un detalle, pero no insignificante, que este asombroso error se halla comúnmente apoyado en una
citación tergiversada.

Todos hemos oído citar la celebrada frase de Dryden:\nota[][-9em]{John Dryden (1631\enraya1700) fue uno de los grandes poetas británicos. La cita proviene del poema satírico \say{Absalom and Achitophel} en el cual Dryden cuenta la historia bíblica de la rebelión de Absalom contra el Rey David.} \textquote{el gran genio es aliado de la locura}. Pero
Dryden no dijo que el gran genio fuera aliado de la locura. El mismo Dryden era un genio y sabía mejor.

Sería difícil encontrar un hombre más romántico y más sensato. Lo que Dryden dijo, fue esto: \textquote{El gran
sabio está frecuentemente próximo a la locura}, y eso es cierto. Es exclusivamente la gran agilidad
intelectual, la que peligra desequilibrarse. También la gente podría recordar, a qué clase de hombre se
refería Dryden. No se trataba de un visionario ajeno a este mundo como Vaughan o George Herbert.\nota{Herbert y Vaughan fueron dos poetas metafísicos del siglo \textsc{xvii}.}

Hablaba de un cínico hombre de mundo, un escéptico, un diplomático, un político práctico. Esos
hombres, ciertamente están próximos al desequilibrio. Su incesante investigar en el cerebro propio y en el
ajeno, es oficio peligroso. Siempre es peligroso para la mente penetrar la mente. Una persona espiritual
preguntó por qué decíamos \say{loco como un sombrerero}. Una persona más espiritual, podría haber
respondido que el sombrerero es loco, porque debe tomar las medidas de la cabeza humana.

Y si los grandes razonadores con frecuencia son maniáticos, es igualmente exacto que los
maniáticos son grandes razonadores.

Cuando me hallaba embarcado en una controversia con el \say{Clarion}.\nota{El \say{Clarion} era un periódico británico y socialista. Dicha polémica se describe en el libro de Chesterton \say{Controversias Blatchford} Robert Blatchford fue fundador y director del \say{Clarion}.},
sobre el tema de la voluntad
libre, el eficiente escritor señor R.\,B. Suthers dijo, que la voluntad libre, era lunatismo, porque implicaba
acciones inmotivadas, y las acciones del lunático son sin causa. No me ocupo aquí de un lapsus
desastroso para la lógica determinista. Evidentemente, si una acción, aun la acción de un lunático, puede
ser inmotivada, se acaba el determinismo.
Porque si un loco puede interrumpir la cadena de causalidad, también puede interrumpirla un
hombre aunque no sea loco. 

Pero mi objeto es destacar algo más práctico. Tal vez fuera natural que un
moderno marxista ignorara todo lo referente a la voluntad libre. Pero sería ciertamente extraño que un
marxista moderno ignorara todo lo que se refiere a los lunáticos. Lo último que se podría decir de un
lunático, es que sus acciones son inmotivadas. Si algunos actos humanos pudieran ser irreflexivamente
llamados \say{sin motivo}, esos serían los insignificantes actos del hombre cuerdo, que silba al caminar; roza
el césped con su bastón; golpea los talones y se frota las manos. Es el hombre contento el que hace las
cosas inútiles; el hombre enfermo no es bastante fuerte para ser un ocioso.

Esas acciones sin causa y descuidadas, son precisamente las que un loco no podría comprender
nunca, porque el loco (como el determinista) tiene demasiado en cuenta las causas de todo. Esas
actividades huecas, tienen para un loco significado de conspiración. Pensará que rozar el pasto, es atentar
contra la propiedad privada. Pensará que golpear los talones, es una señal convenida con un cómplice. Si
por un instante el loco se volviera descuidado, se volvería cuerdo. Todo el que haya tenido la desgracia de
hablar con gente que se hallara en el corazón o al borde del desequilibrio mental, sabe que su
característica más siniestra, es una horrible lucidez para captar el detalle; una facilidad de conectar entre
sí dos cosas perdidas en su mapa confuso como un laberinto. Si ustedes discuten con un loco, es muy
probable que lleven la peor parte en la discusión; porque en muchas formas, la mente del loco es más ágil
y rápida, al no hallarse trabada por todas las cosas que lleva aparejadas el buen discernimiento. No lo
detiene el sentido del humor o de la caridad o las ya enmudecidas certezas de la experiencia. El loco es
más lógico, por carecer de ciertas afecciones de la cordura. La frase común que se aplica a la insania,
desde este punto de vista es errónea. El loco no es el hombre que ha perdido la razón. Loco es el hombre
que ha perdido todo, menos la razón.

Las explicaciones que un loco da sobre algo son completas y con frecuencia, en un sentido
estrictamente racional, hasta son satisfactorias.
O para hablar con más precisión, la explicación del insano si bien no es concluyente, es por lo
menos irrefutable; y esto puede observarse en los dos o tres casos más comunes de locura.

Si un hombre dice (por ejemplo) que los hombres conspiran contra él, no se le puede discutir más
que diciendo que todos los hombres niegan ser conspiradores; que es exactamente lo que harían los
conspiradores. Su exposición concuerda con los hechos tanto como la de ustedes. O si un hombre dice
que es el legítimo Rey de Inglaterra, no es una respuesta adecuada decirle que las autoridades lo catalogan
loco; porque si realmente fuera Rey legítimo de Inglaterra, eso posiblemente sería lo más sabio que
atinaran a hacer las autoridades existentes. O si un hombre dice que es Jesucristo, no es una respuesta
decirle que el mundo niega su divinidad; porque el mundo niega también la divinidad de Cristo.

Sin embargo, ese hombre está equivocado. Pero si intentamos exponer su error en términos exactos,
veremos que no es tan fácil como pudimos suponerle. Tal vez lo más aproximado que podríamos hacer, es
decir esto: que su mente actúa en un círculo perfecto pero estrecho. Un círculo pequeño es tan infinito
como uno grande; pero a pesar de ser tan infinito, no es tan amplio. Del mismo modo, la explicación del
insano es tan completa como la del sano, pero no tan vasta. Una bala es redonda como el mundo, pero no
es el mundo.

Hay algo así como una amplia universalidad; y algo así como una estrecha y restringida eternidad.
Lo podemos ver en muchas religiones modernas.

Ahora, hablando externa y empíricamente, podemos decir que la más consistente e inconfundible
seña de locura, es esta combinación entre la integridad lógica y la contracción espiritual. La teoría del
lunático, explica un vasto número de cosas, pero no explica esas cosas en forma vasta. Quiero decir que si
ustedes, o yo lidiáramos con una mente que se vuelve mórbida, lo indicado sería, no tanto ofrecerle
argumentos como darle aire, para convencerla de que existe algo más limpio y fresco, fuera de la
sofocación de un único argumento. Supongamos que fuera, por ejemplo, el primer caso típico que
mencioné: el caso de un hombre que acusara a todo el mundo de conspirar contra él. Si pudiéramos
expresar nuestros profundos sentimientos de protesta, apelando contra tal obsesión, supongo que le
diríamos algo así: \textquote{Oh; admito que usted tiene su caso y que lo siente de corazón, y que muchas cosas son
como usted dice. Admito que su explicación explica muchas cosas, pero ¡cuántas cosas no explica! ¿No
hay en el mundo más historia que la suya; y todos los hombres se ocupan de usted? Suponga que demos
por sabido los detalles; tal vez cuando aquel hombre en la calle se hizo el que no lo veía, fue por astucia;
tal vez si el agente le preguntó su nombre, lo hizo porque ya lo sabía. Pero ¡cuánto más contento estaría si
le constara que esa gente no se ocupa en absoluto de usted! ¡Cuánto más grande sería su vida si usted se
empequeñeciera en ella! ¡Si pudiera mirar a los otros hombres con curiosidad y gusto comunes, si pudiera
verlos paseando como pasean su radiante egoísmo y su varonil indiferencia! Comenzarían a interesarlo
porque vería que no se interesan en usted. Se evadiría de ese teatro vistoso y mezquino en el que siempre
se representa su dramita personal, y se encontraría bajo un cielo más despejado, en una calle llena de
espléndidos desconocidos.}

O supongamos que fuera el segundo caso de locura, el del hombre que reclama la corona; el
impulso de ustedes, sería contestarle: \textquote{Está bien; tal vez usted sepa que es el Rey de Inglaterra pero, ¿por
qué se preocupa? Haga un esfuerzo magnífico, sea un ser humano y mire de arriba a todos los reyes de la
tierra.}

O podría ser el tercer caso, del loco que se cree Cristo. Si dijéramos lo que sentimos, diríamos:
\textquote{¡Así que usted es el Creador y el Redentor del mundo! ¡Pero qué mundo pequeño debe ser! Qué cielo
más pequeño debe habitar con ángeles no tan grandes como mariposas. ¡Qué aburrido ser Dios! ¡Y un
Dios inadecuado! Realmente, no hay vida más plena ni amor más maravilloso que el suyo; y en realidad ¿es en su
mezquina y penosa compasión que toda carne debe depositar su fe? ¡Cuánto más feliz sería si la masa de
un Dios más grande, pudiera deshacer su pequeño cosmos, desparramara las estrellas como si fueran
pajas, y lo dejara en la inmensidad abierta, libre como otros hombres de mirar hacia arriba y hacia abajo!}

Y hay que recordar que la ciencia más puramente práctica, ataca desde este punto de vista al mal
mental; no intenta discutirlo como una herejía sino simplemente quebrarlo como un encantamiento. Ni la
ciencia moderna ni la religión antigua creen en la completa libertad del pensamiento. La teología reprime
ciertos pensamientos que llama blasfemos. La ciencia reprime ciertos pensamientos que llama morbosos.

Por ejemplo, algunas sociedades religiosas, más o menos exitosamente quieren alejar al hombre del
pensamiento sexual. La nueva sociedad científica, intenta alejarlo del pensamiento de la muerte; que es un
hecho, pero es considerado como un hecho morboso.

Y atendiendo a aquellos cuya morbosidad tiene un dejo de manía, la ciencia moderna se preocupa
de la lógica, mucho menos que un derviche en pleno baile. En esos casos, no es suficiente que el hombre
desgraciado desee la verdad; debe desear la salud. Nada puede salvarlo excepto una ciega ansiedad de
normalidad. Ningún hombre debe creerse a salvo del desequilibrio mental; porque es el órgano que actúa
el pensamiento el que se vuelve enfermo; ingobernable, como si fuera independiente. Sólo puede salvarlo
la voluntad o la fe. Desde que empieza a actuar su razón, actúa en la antigua ruta circular; girará en torno
de su círculo lógico, igual que un hombre en un coche de tercera clase de Juner Circle, girará en torno de
Juner Circle, hasta que realice el voluntario, vigoroso y místico acto, de bajarse en Gower Street. Aquí, la
decisión lo es todo; una puerta debe cerrarse para siempre. Cada remedio, es un remedio desesperado.

Cada cura, es una cura milagrosa. Curar a un hombre no es discutir con un filósofo, es arrojar un
demonio. Y por apaciblemente que trabajen en el asunto los doctores y los filósofos, su actitud es
profundamente incomprensiva. Su actitud es ésta: que el hombre debe dejar de pensar, si quiere seguir
viviendo. Tal tratamiento, es una amputación intelectual.

Si tu cabeza te perturba, córtatela; porque es mejor entrar al Reino de los Cielos no solamente como
un niño sino como un imbécil, que ser arrojado con la inteligencia al infierno\ldots\ o a \say{Hanwell}.

Tal es el loco de los experimentos. Por lo general es un razonador; y con frecuencia un razonador
acertado. Sin duda se le podrá derrotar en un terreno puramente racional planteándole su caso con lógica.

Pero se le puede plantear con mayor precisión en términos más generales y aún más estéticos. Está
encerrado en la pulcra y lúcida prisión de una sola idea; se ha aguzado hasta un penoso extremo. Carece
de la indecisión del sano y de su complejidad. Ahora, según expliqué en la introducción, me propongo
ofrecer en estos primeros capítulos, no tanto el diagrama de una doctrina, cuanto algunas imágenes de un
punto de vista. Y he sido extenso describiendo mi visión del maniático, por esta razón: porque así como
me impresiona el maniático, así me impresionan muchos pensadores modernos.

Esa inconfundible nota que me llega de \say{Hanwell}, la escucho también de muchas cátedras de
Ciencia y de muchas aulas de hoy día; y muchos médicos de alienados, tienen de alienados algo más que
su especialidad.

En todos se manifiesta esa combinación que hemos notado: la combinación de una razón expansiva
y extenuante, con un sentido común contraído y restringido. Son universales en cuanto se aferran a una
explicación razonable y la llevan hasta muy lejos. Pero una muestra, puede prolongarse hasta siempre y
ser no obstante, una pequeña muestra.

En un tablero de ajedrez, ven el blanco sobre el negro; si el universo entero está pavimentado como
el tablero, siempre siguen viendo el blanco sobre el negro. Como el lunático, no pueden alterar su punto
de vista; no pueden hacer un esfuerzo mental y repentinamente verlo negro sobre blanco.

Tomen primero el caso más obvio del materialismo. Para dar una explicación del mundo, el
materialismo tiene una especie de simplicidad insana.

Tiene justo la cualidad del argumento del loco; nos hace sentir simultáneamente, que todo lo abarca
y que todo lo deja afuera.

Contemplen un materialista sincero v eficiente como por ejemplo Mac Cabe, y tendrán esa exacta y
exclusiva sensación. Lo comprende todo; y todo parece no merecer la pena de ser comprendido.

Sus cosmos, puede ser completo en cada remache y en cada engranaje, pero aún así, su cosmos es
más pequeño que nuestro mundo. En cierta forma, su plan, como el lúcido plan del loco, parece insensible
a las remotas energías y a la completa indiferencia de la tierra; es no pensar en las realidades de la tierra:
en los pueblos que luchan, en las madres, en el primer amor y en el terror extendido sobre el mar. La
tierra es tan vasta y el cosmos tan pequeño. El cosmos es, algo así como el agujero más pequeño en el
cual un hombre puede esconder su cabeza.

Hay que entender que ahora no discuto la relación de esas creencias con la verdad, sino, por el
momento, solamente sus relaciones con la salud. Más compenetrados con el argumento, espero atacar el
punto de la verdad objetiva; aquí sólo hablo de un fenómeno psicológico.

Por ahora, no intento probarle a Haeckel que el materialismo es falso, como no intenté probarle al
hombre que se creía Cristo, que elaboraba sobre una creencia errónea. Aquí, destaco exclusivamente el
hecho de que ambos casos son en un mismo sentido completos, y en un mismo sentido incompletos. Se
puede explicar que la indiferencia pública detiene en \say{Hanwell} a un hombre diciendo que esa detención
es la crucifixión de un Dios que el mundo no merecía. La explicación explica. Similarmente se puede
explicar el orden universal, diciendo que todas las cosas, aún las almas de los hombres, son hojas
inevitablemente distribuidas en un árbol por completo inconsciente, i ciego destino de la materia. La
explicación explica, a pesar de no explicar tan completamente como la explicación del loco. Pero aquí el
asunto es que la mente humana normal, no sólo objeta a ambas explicaciones, sino que tiene para las dos
la misma objeción. Su testimonio aproximado es éste: que si el hombre de Hanwell es el verdadero Dios,
no tiene aspecto de serlo. Y similarmente, que si el cosmos del materialista es el verdadero cosmos no
tiene aspecto de cosmos. La cosa se empequeñece. La deidad es menos divina que varios hombres; y
(según Haeckel) el conjunto de la vida, es algo mucho más trivial, gris y estrecho, que varios aspectos
aislados de ella. Las partes parecen mayor que el todo. Porque debemos recordar que la filosofía
materialista, sea o no sea verdadera, tiene por cierto muchas más limitaciones que cualquier religión. En
un sentido por supuesto, todas las ideas inteligentes son limitadas. No pueden ser más vastas que sí
mismas. Un Cristiano está restringido solamente en el sentido en que está restringido un ateo. No puede
pensar que el Cristianismo es falso y seguir siendo cristiano; y el ateo no puede pensar que el ateísmo es
falso y seguir siendo ateo.

Pero siendo las cosas como son, hay un aspecto especial en el cual el materialismo tiene más
restricciones que el espiritualismo. El señor Mac Cabe piensa que soy un esclavo porque no me es
permitido creer en el determinismo. Creo que el señor Mac Cabe es un esclavo porque no le está
permitido creer en las hadas. Pero si examinamos las dos prohibiciones, veremos que la suya es mucho
más absoluta que la mía. El cristiano es muy libre de creer que en el mundo hay un conjunto de
ordenamientos establecidos y de sucesos inevitables. Pero el materialista no puede aceptar ni el más
mínimo dejo de espiritualismo o de milagro.

Al pobre señor Mac Cabe, no le está permitido admitir la posibilidad de que exista un geniecillo ni
escondido en una flor. El cristiano admite que el universo es variado y aún mezclado, tal como el hombre
cuerdo admite su propia complejidad. El hombre cuerdo sabe que tiene un poco de bestia, un poco de
demonio, un poco de santo y un poco de ciudadano. Y lo que es más, el hombre realmente cuerdo, admite
tener, sabe que tiene, algo de loco. Pero el mundo materialista es muy sólido y simple, así como el loco,
está completamente seguro de ser cuerdo. El materialista está seguro de que la historia es simplemente y
solamente una cadena de casualidades, así como la interesante persona que se mencionó antes, está segura
de ser simplemente y solamente un pollo. Los materialistas y los locos, nunca tienen dudas.

Las doctrinas espirituales, actualmente no limitan la mente tanto como las negaciones materialistas.

Aun creyendo en la inmortalidad, no necesito pensar en ella. Pero si no creo en la inmortalidad, no debo
pensar en ella. En el primer caso la ruta está abierta y puedo llegar tan lejos como quiera. En el segundo,
la ruta está cerrada.

Pero el caso es aún más concluyente y el paralelo con la locura, más extraño todavía. Porque
nuestro caso era contra la agotadora y lógica teoría del lunático, que bien o mal, destruía gradualmente su
humanidad. Ahora el cargo es contra las principales deducciones del materialista, que bien o mal,
gradualmente destruyen su humanidad; no me refiero sólo a la bondad, me refiero a la esperanza, al valor,
a la poesía, a la iniciativa y a todo lo que es humano. Siendo el materialismo lo que conduce al hombre
hacia el fatalismo completo (como generalmente ocurre) es inútil pretender que se trate en ningún sentido
de una fuerza libertadora. Es absurdo decir que avanza especialmente la liberación, cuando el libre
pensamiento sólo se usa para destruir la voluntad libre. Los deterministas atan, no desatan.

A su ley, bien pueden llamarla \say{cadena} de causalidad. Es la peor cadena que puede aprisionar al
ser humano. Si gustan, pueden usar el lenguaje de la libertad en la enseñanza materialista, pero salta a la
vista que tal lenguaje es en ese uso tan, pueden decir que el hombre empleara para conversar con el
hombre encerrado en el manicomio. Si gustan, pueden decir que el hombre es libre de pensar que es un
huevo hervido. Pero seguramente es de más peso e importancia el hecho, de que si es un huevo hervido,
no es libre de comer, beber, dormir, caminar o fumarse un cigarrillo.

Si gustan, igualmente pueden decir que el audaz especulador determinista es libre de no creer en la
voluntad libre. Pero en tal caso, es de mucho más peso e importancia el hecho, de que no es libre de
alabar, de maldecir, de agradecer, de justificar, de discutir, de castigar, de resistir a la tentación, de agitar
muchedumbres, de perdonar pecadores, de reprimir tiranos, o aún de decir \textquote{gracias, por la mostaza}.
Pasando de este asunto, puedo destacar que existe una extraña mistificación que presenta al
fatalismo materialista en cierta manera favorable al perdón, a la abolición de los castigos crueles o de
cualquier clase de castigo. Esto es sorprendentemente opuesto a la verdad. Es muy admisible que la
doctrina necesitarista no hace diferencias, que deja azotando al que azota y al buen amigo exhortando
como antes. Pero evidentemente que si algo detuviera, detendría la exhortación. Que los pecados sean
inevitables, no es un hecho que impide el castigo; si algo impide es la persuasión. Es tan probable que el
determinismo conduzca a la crueldad como a la cobardía. El determinismo no es incompatible con el
hecho de tratar cruelmente a los criminales. Con lo que tal vez es incompatible es con darles tratamiento
benigno; con apelar a sus mejores sentimientos; con alentarlos en su lucha moral.

El determinismo no cree en la eficacia de apelar a la voluntad, pero cree en la eficacia de cambiar el
medio ambiente. No dirá al pecador: \textquote{Ve, y no peques más}, porque el pecador no puede rehuir la ofensa.

Pero puede sumergirlo en aceite hirviendo; porque el aceite hirviendo es un medio ambiente. De ahí que
el materialista, considerado como una silueta, tenga los contornos fantásticos de la silueta del loco.

Ambos asumen una actitud, al mismo tiempo inapelable e intolerable.

Por supuesto, que todo lo que antecede se puede decir no sólo del materialista. Lo mismo podría
aplicarse al extremo opuesto de la lógica especulativa. Hay un escéptico mucho más terrible que el que
cree que todo comenzó en la materia. Es posible encontrar el escéptico que cree que todo comienza en sí
mismo. Él no duda de la existencia de los ángeles o de los demonios, sino de la de los hombres y las
vacas. Para ése, sus propios amigos no son sino una mitología hecha por él. Creó su propio padre y su
propia madre. Esta fantasía horrenda tiene algo decididamente atrayente para el egoísmo en cierta forma
místico de nuestros días. Aquel editor que pensaba que los hombres que se tenían fe, llegarían; esos
buscadores del superhombre que siempre creen encontrarlo mirándose al espejo; esos escritores que
hablan de imprimir su personalidad en vez de crear vida para el mundo, toda esa gente está realmente a
una cuarta de aquella vaciedad horrible.

Entonces, cuando en torno al hombre el mundo se haya oscurecido como una mentira, cuando los
amigos se desvanezca en espíritus y vacilen los cimientos de la tierra; entonces, cuando no creyendo en
nada y en nadie el hombre se encuentre a solas en su pesadilla, entonces el gran lema individualista se
trazará sobre él como una ironía vengadora. Las estrellas apenas serán puntos en la oscuridad de su propio
cerebro; el rostro de la madre sólo será un ensayo de su lápiz loco en las paredes del calabozo. Pero sobre
la puerta de su celda se habrá escrito con horrible verdad: \say{Cree en sí mismo.}
No obstante, lo única que nos concierne aquí, es destacar que este extremo del pensamiento egoísta,
encierra y exhibe la misma paradoja que el otro extremo del materialismo. En teoría es igualmente
completo e igualmente lisiado en la práctica. En bien de la claridad, es más fácil exponer la idea diciendo
que un hombre puede creer que siempre vive en un sueño. Pero evidentemente no se le puede ofrecer una
prueba positiva de que no sueña por la sencilla razón de que no hay prueba que no se le pueda ofrecer
igualmente mientras está soñando. Pero si el hombre comienza a incendiar Londres y dice que el ama de
llaves pronto lo llamará a tomar el desayuno, lo tomaríamos y lo llevaríamos con otros lógicos a un lugar
que se ha mencionado con frecuencia en el transcurso de este capítulo.

El hombre que no puede creer a sus sentidos y el hombre que no puede creer en nada, son
igualmente insanos, pero no es posible probar el desequilibrio por un error de sus argumentos sino por la
manifiesta equivocación conjunta de sus vidas. Ambos se han encerrado en sendas cajas pintadas
interiormente con el sol y las estrellas; los dos son incapaces de salir, uno a la salud y la dicha del cielo,
otro a la salud y la dicha de la tierra. Su posición es muy razonable; y aún más, en cierto sentido es
infinitamente razonable, así como una moneda de diez centavos es infinitamente redonda. Pero hay algo
así como una infinidad mezquina, una humillada y esclavizada eternidad. Es entretenido advertir que
muchos místicos o escépticos modernos, han tomado como insignia un símbolo oriental, que es muy el
símbolo de esta nulidad extrema. Representan la eternidad por una serpiente con la cola en la boca. Hay
un admirable sarcasmo en esta imagen de una comida poco satisfactoria. La eternidad del materialismo
fatalista, la eternidad de los teósofos arrogantes y de los científicos encumbrados de hoy, está bien
representada por la serpiente que se come la cola; un animal degradado que destruye hasta su propio ser.

Este capítulo es puramente práctico y se refiere al principal signo y elemento actual de la insania,
que es, en resumen, la razón usada sin base; la razón en el vacío. El hombre que comienza a pensar sin la
base de un primer principio adecuado, enloquece; es el hombre que empieza por el mal lado. Y en las
páginas que siguen tenemos que tratar de descubrir cuál es el buen extremo. Pero podemos preguntar, a
guisa de conclusión, si esto es lo que vuelve loro al hombre ¿qué es lo que lo conserva cuerdo?
Hacia el fin de este libro espero dar una respuesta concluyente (algunos pensarán que una respuesta
demasiado concluyente). Mas por el momento, y en la misma forma netamente práctica, es posible dar
una respuesta referente a lo que en la actual historia de la humanidad, puede conservar cuerdos a los
hombres. Mientras tienen misterios, tienen salud; cuando se destruye el misterio, se crea la morbosidad.

El hombre común siempre ha sido cuerdo, porque el hombre común siempre ha sido místico. Siempre ha
aceptado la nebulosidad. Siempre ha tenido un pie en la tierra y otro en el país de las hadas. Siempre ha
conservado la libertad de dudar de sus dioses; pero (contrariamente a los agnósticos de hoy) también ha
conservado su libertad de creer en ellos. Siempre se ha preocupado más de la verdad que de la
consistencia. Si vio dos verdades que se contradecían mutuamente, tomó las verdades y la contradicción
junto con ellas. Su vista espiritual es estereoscópica, como su vista física. Al mismo tiempo ve dos cosas
diferentes, y no obstante, o por lo mismo, las ve mejor.

De ahí que siempre haya existido algo como el destino, pero también algo como la libertad de
albedrío. De ahí que creyó que de los niños era el reino de los cielos, y que no obstante lo cual, debían
obedecer en el reino de la tierra. Admiró a la juventud porque era joven y a la vejez porque no lo era.

Es, precisamente este don de asociar las aparentes contradicciones, lo que constituye toda la
elasticidad del hombre sano. El único secreto del misticismo es éste: que el hombre puede entenderlo todo
merced a la ayuda de todo lo que no entiende. El lógico mórbido, intenta dilucidarlo todo y sólo consigue
volverlo todo misterio. El místico permite que algo sea misterioso, y todo lo demás se vuelve lúcido. El
determinista hace muy clara la teoría de causalidad y luego descubre que no puede decir \say{por favor} a la
mucama. El Cristiano acepta que la libertad de albedrío siga siendo un misterio sagrado; por eso sus
relaciones con la mucama son de una cristalina y luminosa claridad. Pone la simiente del dogma en una
oscuridad central; pero la simiente germina y se ramifica en todas direcciones con espontánea y saludable
abundancia. Así como hemos tomado al círculo como símbolo de la razón y de la locura, muy bien
podemos tomar a la cruz como símbolo al mismo tiempo de la salud y del misterio. El budismo es
centrípeto pero el Cristianismo centrífugo: se vuelca hacia afuera. Porque el círculo es perfecto e infinito
en su naturaleza; pero se halla siempre limitado a su tamaño; nunca puede ser mayor ni más pequeño.

Pero la cruz, pese a tener en su centro una fusión y una contradicción, puede prolongar hasta siempre sus
cuatro brazos, sin alterar su estructura.

Puede agrandarse sin cambiar nunca, porque en su centro yace una paradoja. El círculo vuelve sobre
sí mismo y está cernido.

La cruz abre sus brazos a los cuatro vientos; es el indicador de los viajeros libres.

Hablando de este profundo tema, los símbolos escuetos son de un valor confuso; y otro símbolo
tomado de la naturaleza expresará con claridad suficiente, lo que es el misticismo para la raza humana.

La única cosa creada que no podemos ver, es la única cosa a cuya luz podemos verlo todo. Como el
sol en su ocaso, el misticismo explica todo lo demás con los rayos de su invisibilidad victoriosa.

El intelectualismo desinteresado es (en el exacto sentido del dicho popular) puro brillo de luna,
porque es luz sin calor y luz reflejada de un mundo muerto. Pero los griegos tenían razón cuando hicieron
a Apolo dios de la imaginación y de la sensatez. Luego hablaré de los dogmas de necesidad y de un credo
especial. Pero ese trascendentalismo según el cual viven los hombres, originariamente tiene mucho de la
posición del sol en el cielo. Tenemos conciencia de él, como de una especie.
\finalCapituloOrnamento
