% Clase del documento y tamaño de letra
\documentclass[draft,10pt,letterpaper,twocolumn,fleqn,leqno,openany]{book}
% Margenes
\usepackage[top=3.5cm,bottom=4cm,inner=2.5cm,outer=2.5cm]{geometry}
%\usepackage[top=2cm,bottom=2cm,left=1cm,right=9cm,headsep=1.4em,marginparwidth=8cm]{geometry}
%\usepackage[top=2cm,bottom=2cm,left=5.3cm,right=5.3cm,headsep=1.4em]{geometry}
% Idioma Español
\usepackage[spanish,es-noindentfirst]{babel}
% Fuentes: Charter, Fira sans y mono. Newtx para mate. Tambien Source serif pro para los títulos.
\usepackage{amsmath}
\usepackage[T1]{fontenc}
\usepackage[rm={lining=true}]{cfr-lm}
\usepackage[scale=.9,light]{plex-sans}
\usepackage[scale=.87]{plex-mono}
\usepackage[scr=dutchcal,cal=euler,bb=fourier]{mathalfa}
\usepackage{bm}

\newcommand{\textord}[1]{\textsuperscript{\hspace*{-2pt}\normalfont\scshape #1}}

\linespread{1.05}
\frenchspacing
\hfuzz1pc
\usepackage[final,tracking=smallcaps,expansion=alltext,protrusion=true]{microtype}
\SetTracking{encoding=*,shape=sc}{50}
% Paquetes de matematicas (Si hay mate en el documento, se puede comentar esta linea junto con las defiiciones de teoremas al final)
% Paquetes varios (Todos menos xcolor son necesarios)
\usepackage{titlesec,xcolor,fancyhdr,marginnote,graphicx,float,cancel,tikz,pgfplots,pgf,caption}

\captionsetup{font={footnotesize,sf},margin=3em}

%Listas de código
\usepackage[final]{listings}
\lstset{ 
	frame = leftline,
	rulecolor=\color{grei},
	framerule=.3em,
	basicstyle=\ttfamily,
	framexleftmargin=5pt,
	xleftmargin=1.3em,
}

% Listas mejores
\renewcommand{\theenumi}{\Alph{enumi}}

% Comando formato de las cabezeras y pies de página
\renewcommand{\headrulewidth}{0pt}

\newcommand{\cabe}[2]{
	\pagestyle{fancy}
	\fancyhf{}
	\chead{\sffamily  #1}
	\rfoot{\sffamily\thepage} 
	\lfoot{\sffamily #2}
	\setlength{\headheight}{14pt}
}

	\fancypagestyle{plain}{
	\fancyhf{}
	\rhead{}
	\rfoot{\sffamily\thepage }
	\lhead{}
	\lfoot{\sffamily Noviembre, 2019}
	}

\makeatletter%
\long\def\@makefnmark{%
	\hbox {{\sffamily [\@thefnmark] }}}%
\makeatother

\makeatletter%
\long\def\@makefntext#1{%
	\parindent 1em\noindent\sffamily \hb@xt@ 1.8em{\hss \@makefnmark}#1}%
\makeatother

% Formato de las secciones
\newcommand{\HUGE}{\fontsize{40}{40}\selectfont}
\newcommand{\dispSize}{\fontsize{18}{18}\selectfont}
\newcommand{\secSize}{\fontsize{11}{11}\selectfont}

\titleformat{\chapter}[hang]
	{\flushleft\dispSize}
	{\normalsize Algebra 3}
	{1em}
	{}
	[]
\titlespacing*{\chapter}
	{0em}
	{0em}
	{1.5em}
\titleformat{\section}[hang]
	{\raggedright\secSize\itshape}
	{\thesection º}
	{0.5em}
	{}
	[]
\titlespacing*{\section}
	{0em}
	{1em}
	{1em}
\titleformat{\subsection}
	{\flushleft\itshape}
	{\S\hspace{1pt}\thesubsection}
	{.5em}
	{}
	[]
\titlespacing*{\subsection}
	{0em}
	{1em}
	{0em}
\titleformat{\paragraph}[runin]
	{\scshape}
	{}
	{0em}
	{}
	[]
	
	% Operators
	\DeclareMathOperator{\Rea}{Re}
	\DeclareMathOperator{\Ima}{Im}
	\DeclareMathOperator{\car}{car}
	\DeclareMathOperator{\traz}{tr}
	\DeclareMathOperator{\gen}{gen}
	\DeclareMathOperator{\mcm}{mcm}
	\DeclareMathOperator{\id}{id}
%%%%%%%%%%%%%%%%%%%%%%%%%%%%%%%%%%%%%%%%%%%%%%%%%%%%%%%%%%%%%%%%%%%%%%%%%%%%%%%%%%%%%%%%%%%%%%%%%%%%
\usepackage{ntheorem}
% Formato de los Ejercicios
\theoremstyle{nonumberplain}
\theoremindent0cm
\theorembodyfont{\itshape}
\theoremseparator{.}
\theoremheaderfont{\upshape\scshape}
\newtheorem{teo}{teorema}
\newtheorem{ejer}{Ejercicio}
% Formato de las Soluciones
\theoremstyle{nonumberplain}
\theoremindent0cm
\theoremheaderfont{\itshape}
\theorembodyfont{\normalfont}
\theoremseparator{.}
\newtheorem{sol}{Solución}
\theoremheaderfont{\scshape}
\theoremsymbol{\ensuremath{\qed}}
\newtheorem{proof}{demostración}

\newcommand{\titulo}[2]{
	\thispagestyle{plain}
	\twocolumn[{%
		\begin{flushleft}\selectfont
			\textsc{#1} \\[.4em]
			{\titlefont\fontsize{15}{15}\selectfont #2} \\[.2em]
			\textsc{Jhonny Lanzuisi}, 1510759
		\end{flushleft}
	}]
}
\newcommand{\tituloD}[2]{
	\thispagestyle{plain}
	\begin{flushleft}\selectfont
		\textsc{#1} \\[.4em]
		{\titlefont\fontsize{15}{15}\selectfont #2} \\[.2em]
		\textsc{Jhonny Lanzuisi}, 1510759
	\end{flushleft}
	%	\vspace*{2em}
}
% Atajos
%\renewcommand{\emptyset}{\varnothing}
%\newcommand{\id}[1]{\symfrak{#1}}
\newcommand{\inte}[4]{\int_{#1}^{#2} #3\, \mathrm{d}#4}
%\newcommand{\an}[1]{\symsfup{#1}}
\newcommand{\grd}[1]{\nabla #1} % Gradientes
\newcommand{\li}[1]{\lim\limits_{#1}} % Limites
\newcommand{\inv}{^{-1}} % Inversos multiplicativos
\newcommand{\ort}{^\perp}
\newcommand{\pd}[2]{\dfrac{\partial #1}{\partial #2}} % Derivadas parciales
\newcommand{\tri}{\mathbin{\triangle}}
\newcommand{\Px}{\mathcal{P}(X)}
\renewcommand{\t}{\times}
\newcommand{\pds}[2]{\dfrac{\partial^2 #1}{\partial #2^2}}
\newcommand{\pdx}[3]{\dfrac{\partial^2 #1}{\partial #2 \partial #3}}
\newcommand{\Zn}{\mathcal{Z}/n\mathcal{Z}}
\newcommand{\Rn}{\mathcal{R}^n}
\newcommand{\Fn}{\mathcal{F}^n}
\newcommand{\F}{\mathcal{F}}
\newcommand{\V}{\mathcal{V}}
\newcommand{\Cn}{\mathcal{C}^n}
\newcommand{\Rm}{\mathcal{R}^m}
\newcommand{\R}{\mathcal{R}}
\newcommand{\Co}{\mathcal{C}}
\newcommand{\Z}{\mathcal{Z}}
\newcommand{\Q}{\mathcal{Q}}
\newcommand{\N}{\mathcal{N}}
\newcommand{\Ha}{\mathcal{H}}
\newcommand{\eb}[1]{\left\{ #1 \right\}}
\newcommand{\norm}{\lVert\phantom{a}\rVert}
\newcommand{\Norm}[1]{\left\lVert#1\right\rVert}
\newcommand{\iprod}{\langle\phantom{x}\, ,\phantom{x}\rangle}
\newcommand{\Iprod}[2]{\left\langle#1,#2\right\rangle}
\newcommand{\Bform}{(\phantom{a},\phantom{a})}
\newcommand{\bform}[2]{(#1,#2)}
\newcommand{\Ve}[1]{\bm{#1}}
\newcommand{\ve}[1]{\mathcal{#1}}
\newcommand{\zv}[1]{Z(v_{#1},T)}
\newcommand{\proy}[2]{\mathrm{Proy}_{#1}(#2)}