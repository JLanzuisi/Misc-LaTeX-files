% !TeX program = XeLaTeX
% Clase del documento y tamaño de letra
\documentclass[draft,11pt,letterpaper]{article}
% Callate acerca de los fontaxes, LaTeX.
% UTF-8
% Margenes
\usepackage[top=2.5cm,bottom=3cm,left=5cm,right=5cm,headsep=1.4em,marginpar=2cm]{geometry}
% Idioma
\usepackage[spanish,es-nodecimaldot]{babel}
%\usepackage[utf8]{inputenc}
% Fuentes: Charter, Fira sans y mono. Newtx para mate. Tambien Source serif pro para los títulos.
\usepackage{amsmath}
\usepackage{unicode-math}
\setmathfont{XITSMath-Regular}[
Path=/home/jhonny/.fonts/,
Extension=.otf,
Scale=MatchLowercase
]
\setmathfont{TeX Gyre Termes Math}[
Scale=MatchLowercase,
range={\mathit/{Latin},\sum},
]
\setmathfont{Garamond-Math}[
Path=/home/jhonny/.fonts/,
Extension=.otf,
Scale=MatchLowercase,
range={\in,\notin},
]
\setmathfont{Brill-Bold}[
Path=/home/jhonny/.fonts/,
Extension=.ttf,
range=\mathbfup
]
\setmainfont{Brill}[
Path=/home/jhonny/.fonts/,
Extension=.ttf,
UprightFont=*-Regular,
BoldFont=*-Bold,
ItalicFont=*-Italic,
BoldItalicFont=*-BoldItalic,
]
\setmonofont{luxim}[
Path=/home/jhonny/.fonts/,
Extension=.ttf,
UprightFont=*-Regular,
BoldFont=*-Bold,
ItalicFont=*-Italic,
BoldItalicFont=*-BoldItalic,
Scale=MatchLowercase
]
% Espacio entre lineas
\usepackage{microtype}
% Paquetes de matematicas (Si hay mate en el documento, se puede comentar esta linea)
\usepackage{ntheorem}
% Paquetes varios (Todos menos xcolor son necesarios)
\usepackage{titlesec,fancyhdr,marginnote,graphicx,float,mleftright,dsfont}
%Listas de código
\usepackage[final]{listings}
\lstset{
%	numbers=left, 
%	numberstyle=\small\ttfamily, 
%	numbersep=8pt, 
	frame = leftline,
	rulecolor=\color{grei},
	framerule=.3em,
	basicstyle=\ttfamily,
	framexleftmargin=5pt,
	xleftmargin=1.3em,
}
% Colores
\usepackage[dvipsnames]{xcolor}
\definecolor{wine}{HTML}{722f37}
\definecolor{wine2}{HTML}{c02132}
\definecolor{wine3}{HTML}{66000c}
\definecolor{darkgrei}{HTML}{3a3335}
\definecolor{grei}{HTML}{b7ad99}
\definecolor{peach}{HTML}{ffd9da}
% Formato cabezeras y pies de pagina
\newcommand{\tcabe}[2]{
	\fancypagestyle{plain}{
		
		\fancyhf{}
		\rhead{\itshape\color{darkgrei} #1}
		\rfoot{ \small\thepage }
		\cfoot{ \small\today}
		\lfoot{\small #2}
	}
}
\newcommand{\cabe}[1]{
	\pagestyle{fancy}
	\fancyhf{}
	\renewcommand{\headrulewidth}{0pt}
	\rhead{\color{darkgrei}\itshape #1}
	\rfoot{\small\thepage} 
	\setlength{\headheight}{14pt}
}

\titleformat{\section}
{\large\scshape}
{\thesection}
{1em}
{}
\titlespacing*{\section}
{0em}
{0em}
{.5em}
\newcommand{\mar}[1]{
 \marginnote{\itshape #1}
}

% Entorne 'Enumerate' mas bonito
\usepackage{enumitem}
\setlist[enumerate]{label=\alph*),labelsep=3.5pt,itemsep=0em,wide,labelwidth=!,labelindent=0pt,listparindent=\parindent,parsep=0pt}
% Formato de los Ejercicios
\theoremstyle{plain}
\theoremindent0cm
\theorembodyfont{\normalfont}
\theoremseparator{.}
\theoremheaderfont{\scshape}
\newtheorem{ejer}{Ejercicio}
% Formato de las Soluciones
\theoremstyle{nonumberplain}
\theoremindent0cm
\theoremheaderfont{\itshape}
\theorembodyfont{\normalfont}
\theoremseparator{.}
\newtheorem{sol}{\it Solución}
{
	\theoremstyle{nonumberplain}
	\theoremindent0cm
	\theoremseparator{}
	\theorempreskip{.5em}
	\theorempostskip{.5em}
	\theorembodyfont{\small}
	\theoremheaderfont{\small}
	\newtheorem{nota}{Nota}
}
% Atajos

\newcommand{\mybinom}[2]{%
	\mleft(
	\begin{array}{@{}c@{\,}} #1\\#2 \end{array}
	\mright)}
\newcommand{\grd}[1]{\nabla #1} % Gradientes
\newcommand{\li}[1]{\lim\limits_{#1}} % Limites
\newcommand{\inv}{^{-1}} % Inversos multiplicativos
\newcommand{\pd}[2]{\dfrac{\partial #1}{\partial #2}} % Derivadas parciales
\newcommand\tri{\mathbin{\triangle}}
\newcommand{\Px}{\mathcal{P}(X)}
\renewcommand{\t}{\times}
\newcommand{\pds}[2]{\dfrac{\partial^2 #1}{\partial #2^2}}
\newcommand{\pdx}[3]{\dfrac{\partial^2 #1}{\partial #2 \partial #3}}
\newcommand{\Rn}{\mathbfup{R}^{\mathrm n}}
\newcommand{\R}{\mathbfup{R}}
\newcommand{\Co}{\mathbfup{C}}
\newcommand{\calU}{\mathcal{U}}
\newcommand{\Q}{\mathbfup{Q}}
\newcommand{\Z}{\mathbfup{Z}}
\newcommand{\N}{\mathbfup{N}}
\renewcommand{\(}{\left(}
\renewcommand{\)}{\right)}
\newcommand{\eb}[1]{\left\{ #1 \right\}}
\DeclareMathOperator{\sen}{sen}