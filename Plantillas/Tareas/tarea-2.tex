% !TeX program = XeLaTeX
% Clase del documento y tamaño de letra
\documentclass[draft,10pt,letterpaper,fleqn,leqno]{article}
% Callate acerca de los fontaxes, LaTeX.
% UTF-8
% Margenes
\usepackage[top=2.5cm,bottom=3cm,left=5cm,right=5cm,headsep=1.4em,marginpar=2cm]{geometry}
%\usepackage[top=2cm,bottom=1.5cm,left=2.4cm,right=2.4cm,headsep=1.4em,marginpar=2cm]{geometry}
% Idioma
\usepackage[spanish,es-nodecimaldot]{babel}
\usepackage{enumitem}
\setlist[enumerate]{label=\alph*)}
\usepackage{amsmath}
% Fuentes: Charter, Fira sans y mono. Newtx para mate. Tambien Source serif pro para los títulos.
\usepackage[spanish]{babel}% load before XCharter
\usepackage{unicode-math}
\setmainfont{Stix2Text}[
Path=/home/jhonny/.fonts/STIX/,
Extension=.otf,
UprightFont=*-Regular,
ItalicFont=*-Italic,
BoldFont=*-Bold,
Scale=.98,
Numbers={OldStyle,Proportional},
]
\defaultfontfeatures{Scale=MatchLowercase,Numbers={OldStyle,Proportional}}
\setmonofont{Go Mono}
\newfontfamily{\lmr}{Latin Modern Roman}
\newfontfamily{\lmrd}{Latin Modern Roman Demi}
\setmathfont{STIX Two Math}
\linespread{1.05}
\frenchspacing
\hfuzz1pc
\usepackage[final]{microtype}
% Paquetes de matematicas (Si hay mate en el documento, se puede comentar esta linea)
\usepackage{ntheorem}
% Paquetes varios (Todos menos xcolor son necesarios)
\usepackage{titlesec,fancyhdr,marginnote,graphicx,float,mleftright,bm}
%Listas de código
\usepackage[final]{listings}
\lstset{
%	numbers=left, 
%	numberstyle=\small\ttfamily, 
%	numbersep=8pt, 
	frame = leftline,
	rulecolor=\color{grei},
	framerule=.3em,
	basicstyle=\ttfamily,
	framexleftmargin=5pt,
	xleftmargin=1.3em,
}
% Colores
\usepackage[dvipsnames]{xcolor}
\definecolor{wine}{HTML}{722f37}
\definecolor{wine2}{HTML}{c02132}
\definecolor{wine3}{HTML}{66000c}
\definecolor{darkgrei}{HTML}{3a3335}
\definecolor{grei}{HTML}{b7ad99}
\definecolor{peach}{HTML}{ffd9da}
% Formato cabezeras y pies de pagina
\newcommand{\tcabe}[1]{
	\fancypagestyle{plain}{
		\fancyhf{}
		\lhead{\scshape\addfontfeature{LetterSpace=3.0} #1}
		\rfoot{\thepage }
		\cfoot{\today}
	}
}
\newcommand{\cabe}[1]{
	\pagestyle{fancy}
	\fancyhf{}
	\renewcommand{\headrulewidth}{0pt}
	\rhead{\itshape #1}
	\rfoot{\small\thepage} 
	\setlength{\headheight}{14pt}
}
\titleformat{\section}
{\raggedleft\LARGE\lmrd}
{}
{0em}
{\hrule}
[\hrule]
\titlespacing*{\section}
{0em}
{0em}
{3em}
\titleformat{\subsection}
{\lmr\bfseries\centering}
{\thesubsection}
{0em}
{} 
[]
\titlespacing*{\subsection}
{0em}
{1em}
{1em}
\newcommand{\mar}[1]{
 \marginnote{\itshape #1}
}
%No indent
\makeatletter
% package indentfirst says \let\@afterindentfalse\@afterindenttrue
% and we revert this modification, reinstating the original definitio
% of \@afterindentfalse
\def\@afterindentfalse{\let\if@afterindent\iffalse}
\makeatother
% Entorne 'Enumerate' mas bonito

% Formato de los Ejercicios
\theoremstyle{plain}
\theoremindent0cm
\theorembodyfont{\normalfont}
\theoremseparator{.}
\theoremheaderfont{\bfseries}
\newtheorem{ejer}{Ejercicio}
\newtheorem{preg}{Pregunta}
% Formato de las Soluciones
\theoremstyle{nonumberplain}
\theoremindent0cm
\theoremheaderfont{\itshape}
\theorembodyfont{\normalfont}
\theoremseparator{.}
\newtheorem{sol}{\it Solución}
\newtheorem{resp}{\it Respuesta}
{
	\theoremstyle{nonumberplain}
	\theoremindent0cm
	\theoremseparator{}
	\theorempreskip{.5em}
	\theorempostskip{.5em}
	\theorembodyfont{\small}
	\theoremheaderfont{\small}
	\newtheorem{nota}{Nota}
}
% Atajos

\newcommand{\mybinom}[2]{%
	\mleft(
	\begin{array}{@{}c@{\,}} #1\\#2 \end{array}
	\mright)}
\newcommand{\grd}[1]{\nabla #1} % Gradientes
\newcommand{\li}[1]{\lim\limits_{#1}} % Limites
\newcommand{\inv}{^{-1}} % Inversos multiplicativos
\newcommand{\pd}[2]{\dfrac{\partial #1}{\partial #2}} % Derivadas parciales
\newcommand\tri{\mathbin{\vartriangle}}
\newcommand{\Px}{\mathcal{P}(X)}
\renewcommand{\t}{\times}
\newcommand{\pds}[2]{\dfrac{\partial^2 #1}{\partial #2^2}}
\newcommand{\pdx}[3]{\dfrac{\partial^2 #1}{\partial #2 \partial #3}}
\newcommand{\Rn}{\mathbfup{R}^n}
\newcommand{\R}{\mathbfup{R}}
\newcommand{\Co}{\mathbfup{C}}
\newcommand{\calU}{\mathcal{U}}
\newcommand{\Q}{\mathbfup{Q}}
\newcommand{\Z}{\mathbfup{Z}}
\newcommand{\N}{\mathbfup{N}}
\renewcommand{\(}{\left(}
\renewcommand{\)}{\right)}
\newcommand{\eb}[1]{\left\{ #1 \right\}}
%\newcommand{\Rea}{\symup{Re}}
%\newcommand{\Ima}{\symup{Im}}
\DeclareMathOperator{\sen}{sen}
\DeclareMathOperator{\Rea}{Re}
\DeclareMathOperator{\Ima}{Im}