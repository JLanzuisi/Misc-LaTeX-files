% Clase del documento y tamano de letra
\documentclass[draft,11pt,letterpaper,openany,fleqn,leqno]{book}
% Márgenes
%\usepackage[top=2cm,bottom=1.5cm,left=2.8cm,right=2.8cm,marginpar=2cm]{geometry}
\usepackage[top=3.5cm,bottom=4cm,left=5cm,right=5cm,marginpar=3.5cm]{geometry}
%\usepackage[left=2cm,right=8cm,asymmetric,marginpar=5.5cm,marginparsep=.7cm,twoside]{geometry} % Usar para doble-lado
% Idioma Español
\usepackage[spanish]{babel}
% UTF8
\usepackage{amsmath}
% Paquetes de mate
% Fuentes: STIX2 para texto y matemáticas, FiraSans para sans serif, FiraMono para monoespaciada, CooperHewitt para títulos.
\usepackage{cfr-lm}
\usepackage[light,scale=.90]{Chivo}
\usepackage[scale=.85]{plex-mono}
\newcommand{\titlefont}{\fontfamily{LibreBodoni-TLF}\selectfont}
\usepackage[bb=fourier,scr=dutchcal,cal=boondoxo,frak=euler]{mathalfa}
\usepackage{bm}
\frenchspacing
\hfuzz1pc
\usepackage[final,tracking=smallcaps,expansion=alltext,protrusion=true]{microtype}
\SetTracking{encoding=*,shape=sc}{50}
% Paquetes varios
\usepackage{marginnote,changepage,graphicx,hyperref,ellipsis,pdfpages,adforn,multicol}
% Bloques de código
\usepackage[final]{listings}
\lstset{
	numberstyle=\small\ttfamily, 
	numbersep=8pt, 
	basicstyle=\ttfamily,
	framexleftmargin=15pt,
	xleftmargin=1.6em,
}
% Colores
\usepackage[]{xcolor}
\definecolor{darkgrei}{HTML}{3a3335}
\definecolor{grei}{HTML}{b7ad99}
\definecolor{peach}{HTML}{ffd9da}
\definecolor{wine}{HTML}{722f37}
\definecolor{wine2}{HTML}{c02132}
\definecolor{wine3}{HTML}{66000c}
\definecolor{wine4}{HTML}{cdd8ff}
% Lista de contenidos
\addto\captionsspanish{
	\renewcommand{\contentsname}
	{\lsstyle ÍNDICE}			% Nombre personalizado de la lista
}
\usepackage{titletoc}
\usepackage{titletoc}
\titlecontents{chapter}
[1.6em]                                   % Margen izquierdo
{\vspace{1em}}%
{\contentsmargin{0pt}\scshape                  % Formato de la entrada%
	\large}
{\contentsmargin{0pt}\Large}            % Formato de entrada sin número
{}                 		
[\vspace{0pt}]						% Paquete para formato personalizado de la lista
\titlecontents{section}
[1.6em]                                   % Margen izquierdo
{\vspace{.7em}}%
{\contentsmargin{0pt}	                   % Formato de la entrada%
}
{\contentsmargin{0pt}\Large}            % Formato de entrada sin número
{\small\contentspage}                 		
[\vspace{6pt}] 
\titlecontents{subsection}					% El formato es el mismo que antes
[1.6em]                                               
{}%
{\contentsmargin{0pt}\itshape                             
	\enspace%
}
{\contentsmargin{0pt}\large}                      
{\small\contentspage}                 
[\vspace{3pt}]

% Comando para el nombre de la sección actual
\usepackage{nameref}
\makeatletter
\newcommand*{\currentname}{\@currentlabelname}
\makeatother
% Formato de secciones y subsecciones
\usepackage[compact]{titlesec}
\titleformat{\chapter}[display]
{\flushleft\titlefont\LARGE\itshape}
{\hspace{-2em} Capítulo\, \upshape\thechapter}
{0.5em}
{}
[]
\titlespacing*{\chapter}
{0em}
{0em}
{1.5em}
\titleformat{\section}[display]
{\flushleft\titlefont\LARGE\itshape}
{\hspace{-2em} Sección\, \upshape\thesection}
{0.5em}
{}
[]
\titlespacing*{\section}
{0em}
{0em}
{1.5em}
\titleformat{\subsection}[hang]
{\centering\bfseries\titlefont}
{\S\thesubsection}
{0.5em}
{}
[]
%	{\hrule\vspace{.15em}}
%	[\vspace{.15em}\hrule]
\titlespacing*{\subsection}
{0em}
{2em}
{1em}
%No indent
\makeatletter
% package indentfirst says \let\@afterindentfalse\@afterindenttrue
% and we revert this modification, reinstating the original definitio
% of \@afterindentfalse
\def\@afterindentfalse{\let\if@afterindent\iffalse}
\makeatother
% Encabezados y pies de página
\usepackage{fancyhdr}
\pagestyle{fancy}
\fancyhf{}
 % Sin línea horizontal
\rhead{\color{darkgrei}\sffamily\currentname}
\rfoot{\sffamily\thepage}
\renewcommand{\headrulewidth}{0pt}
\setlength{\headheight}{14pt}
\fancypagestyle{plain}{% Esto cambia el estilo "plain"
		\fancyhf{}%
		\fancyfoot[R]{}%
}

% Listas mas lindas
%\newcommand\MyRectangle{\rule{.36em}{1.5ex}}
% Entornos de 'teorema'
\usepackage[thmmarks]{ntheorem}
{	% Definiciones
	\theoremstyle{marginbreak}
	\theoremindent0cm
	\theorembodyfont{\normalfont}
	\theoremheaderfont{\bfseries}
	\newtheorem{defi}{Definición}[section]
}
{	% Teoremas, Corolarios, Proposiciones, Lemas
	\theoremstyle{margin}
	\theoremindent0cm
	\theoremseparator{.}
	\theoremheaderfont{\itshape}
	\theorembodyfont{\itshape}
	\newtheorem{teo}{Teorema}[section]
	\newtheorem{cor}{Corolario}[section]
	\newtheorem{prop}{Proposición}[section]
	\newtheorem{lem}{Lema}[section]
	% Ejercicios y Ejemplos
	\theorembodyfont{\normalfont}
	\theoremheaderfont{\normalfont}
	\newtheorem{ejem}{Ejemplo}[section]
	\newtheorem{ejer}{Ejercicio}[section]
}
{	% Soluciones (de los ejericicos), Observaciones y demostraciones
	\theoremstyle{nonumberplain}
	\theoremindent0cm
	\theoremheaderfont{\itshape}
	\theorembodyfont{\normalfont}
	\theoremseparator{.}
	\newtheorem{sol}{\itshape Solución}
	\newtheorem{obs}{\itshape Observación}
	\theoremsymbol{\ensuremath{\qed}}
	\newtheorem{proof}{Demostración}
}
{	% Notas en el texto
	\theoremstyle{nonumberplain}
	\theoremindent0cm
	\theorempreskip{.5em}
	\theorempostskip{.5em}
	\theorembodyfont{\small}
	\theoremheaderfont{\small}
	\newtheorem{nota}{\adforn{33}}
}
\newcommand{\Nota}[1]{\marginnote{#1}}
\renewcommand*{\raggedleftmarginnote}{}
\renewcommand*{\raggedrightmarginnote}{}
% Citas al principio de las secciones
\newcommand{\cita}[2]{
%\pagecolor{wine4}
\begin{center}
	\begin{minipage}{.5\linewidth}\centering\small
		{ ``#1''} \\[1em]
		\scshape\adforn{63} #2
	\end{minipage}
\end{center}
\vspace{8em}
%\nopagecolor
}
% Notas al pie se resetan por seccion
\makeatletter
\@addtoreset{footnote}{section}
\makeatother
% Apéndice
%\newcommand{\apen}{
%	\titleformat{\section}[display]
%	{\raggedright\Huge}
%	{\huge\thesection}
%	{0em}
%	{}
%	[]
%\titlespacing*{\section}
%	{0em}
%	{0em}
%	{1em}
%}
% Macros (Atajos)
%\newcommand{\n}{\mathrm{n}}
%\newcommand{\mr}[1]{\symup{#1}}
%\renewcommand{\emptyset}{\varnothing}
\newcommand{\id}[1]{\mathfrak{#1}}
\newcommand{\an}[1]{\mathscr{#1}}
\newcommand{\subsq}{\subseteq}
\newcommand{\inv}{^{-1}}
\newcommand{\notall}[1]{\marginnote{\small \textcolor{darkgray}{#1}}}
\newcommand{\ho}[2]{\text{hom}(#1,#2)}
\newcommand{\hoo}[1]{\text{hom}(#1)}
\renewcommand{\t}{\times}
\newcommand{\dt}[2]{#1_1 #2 #1_2 #2 \cdots #2 #1_n}
\newcommand{\li}[2]{\lim_{#1 \to #2}}
\newcommand{\pd}[2]{\frac{\partial #1}{\partial #2}}
\newcommand{\pds}[2]{\dfrac{\partial^2 #1}{\partial #2^2}}
\newcommand{\pdx}[3]{\dfrac{\partial^2 #1}{\partial #2 \partial #3}}
\newcommand{\pol}[1]{#1_n x^n + #1_{n-1} x^{n-1} + \dots + #1_1 x + #1_0}
\newcommand{\Zn}{\mathscr{Z}/n\mathscr{Z}}
\newcommand{\Rn}{\mathscr{R}^{\mathrm n}}
\newcommand{\R}{\mathscr{R}}
\newcommand{\Co}{\mathscr{C}}
\newcommand{\Z}{\mathscr{Z}}
\newcommand{\Q}{\mathscr{Q}}
\newcommand{\N}{\mathscr{N}}
\newcommand{\Ha}{\mathscr{H}}
\newcommand{\eb}[1]{\left\{ #1 \right\}}
\DeclareMathOperator{\car}{car}