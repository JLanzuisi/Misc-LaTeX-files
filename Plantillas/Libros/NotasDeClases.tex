% Clase del documento y tamano de letra
\documentclass[draft,10pt,letterpaper,fleqn,twocolumn,openany]{book}
% Márgenes
\usepackage[top=3.5cm,bottom=4cm,inner=2.5cm,outer=2.5cm]{geometry}
% Idioma Español
\usepackage[spanish,es-noindentfirst]{babel}
% Fuentes: Charter, Fira sans y mono. Newtx para mate. Tambien Source serif pro para los títulos.
\usepackage[T1]{fontenc}
\usepackage[lining]{libertinus-type1}
\usepackage[scale=.9]{PublicSans}
\usepackage[scale=.9]{plex-mono}
\RequirePackage[libertine]{newtxmath}
\usepackage{bm}

\newcommand{\textord}[1]{\textsuperscript{\hspace*{-3pt} #1}}

\linespread{1.05}
\frenchspacing
\hfuzz1pc
\usepackage[final,tracking=smallcaps,expansion=alltext,protrusion=true]{microtype}
\SetTracking{encoding=*,shape=sc}{50}
% Paquetes varios
\usepackage{marginnote,changepage,graphicx,pdfpages,booktabs,float,tikz,color,caption,adforn}
\captionsetup{justification=centering,font={footnotesize,sf},margin=.5em}
\usepackage[makeroom]{cancel}
%\usepackage[colorlinks=true,allcolors=wine]{hyperref}
% Bloques de código
\usepackage[final]{listings}
\lstset{
	numberstyle=\small\ttfamily, 
	numbersep=8pt, 
	basicstyle=\ttfamily,
	framexleftmargin=15pt,
	xleftmargin=1.6em,
}
% Colores
\usepackage{xcolor}
\definecolor{darkgrei}{HTML}{2b2b2b}
\definecolor{grei}{HTML}{b7ad99}
\definecolor{peach}{HTML}{ffd9da}
\definecolor{wine}{HTML}{8e0918}
\definecolor{wine2}{HTML}{721620}
\definecolor{wine3}{HTML}{66000c}
\definecolor{wine4}{HTML}{cdd8ff}
% Lista de contenidos

\usepackage{titletoc}
\titlecontents{chapter}
[1.6em]                                   % Margen izquierdo
{\vspace{.3em}}%
{\contentsmargin{0pt}\scshape                  % Formato de la entrada%
	\large}
{\contentsmargin{0pt}\Large}            % Formato de entrada sin número
{}                 		
[\vspace{4pt}]						% Paquete para formato personalizado de la lista
\titlecontents{section}
	[2.5em]                                   % Margen izquierdo
	{}%
	{\contentsmargin{0pt}	                   % Formato de la entrada%
	 }
	{\contentsmargin{0pt}\Large}            % Formato de entrada sin número
	{\small\contentspage}                 		
	[\vspace{5pt}] 
\titlecontents{subsection}					% El formato es el mismo que antes
	[3.5em]                                               
	{\vspace{-4pt}}%
	{\contentsmargin{0pt}\small                             
	 \enspace%
	}
	{\contentsmargin{0pt}\large}                      
	{\small\contentspage}                 
	[\vspace{3pt}]

% Comando para el nombre de la sección actual
\usepackage{nameref}
\makeatletter
\newcommand*{\currentname}{\@currentlabelname}
\makeatother
% Formato de secciones y subsecciones
\usepackage[compact]{titlesec}
\newcommand{\HUGE}{\fontsize{70}{70}\selectfont}
%\newcommand{\dispFont}{\LibertineDisplay}
\newcommand{\dispSize}{\fontsize{20}{20}\selectfont}
\newcommand{\secSize}{\fontsize{12}{12}\selectfont}

\titleformat{\chapter}[hang]
{\flushleft\dispSize\LibertinusDisplay}
{\HUGE\thechapter}
{.1em}
{}
[]
\titlespacing*{\chapter}
{0em}
{0em}
{1.5em}
\titleformat{\section}[hang]
{\raggedright\secSize\itshape}
{\S\hspace{1pt}\thesubsection}
{0.5em}
{}
[]
\titlespacing*{\section}
{0em}
{1em}
{1em}
\titleformat{\subsection}
{\flushleft\itshape}
{\S\hspace{1pt}\thesubsection}
{.5em}
{}
[]
\titleformat{\paragraph}[runin]
{\scshape}
{}
{0em}
{}
[]
% Encabezados y pies de página
\usepackage{fancyhdr}
\pagestyle{fancy}
\fancyhf{}
\rhead{\sffamily\currentname}
\rfoot{\sffamily\thepage}
\renewcommand{\headrulewidth}{0pt}
\setlength{\headheight}{14pt}
\fancypagestyle{plain}{% Esto cambia el estilo "plain"
		\fancyhf{}%
		\fancyfoot[R]{}%
}

\setlength{\tabcolsep}{3pt}

% Entornos de 'teorema'
\usepackage[thmmarks]{ntheorem}
{	% Definiciones
	\theoremstyle{change}
	\theoremindent.5cm
	\theoremheaderfont{\LibertinusSerifOsF\upshape\scshape}
	\theorembodyfont{\normalfont}
	\newtheorem{defi}{definición}[section]
	\newtheorem{teo}{teorema}[section]
	\newtheorem{cor}{corolario}[section]
	\newtheorem{prop}{proposición}[section]
	\newtheorem{lem}{lema}[section]
}
{	% Teoremas, Corolarios, Proposiciones, Lemas
%	\theoremstyle{changebreak}
%	\theoremindent.5cm
%	\theoremseparator{.}
%	\theoremheaderfont{\upshape\scshape}
%	\theorembodyfont{\normalfont}
%	\newtheorem{teo}{teorema}[section]
%	\newtheorem{cor}{corolario}[section]
%	\newtheorem{prop}{proposición}[section]
%	\newtheorem{lem}{lema}[section]
	% Ejercicios y Ejemplos
	\theoremstyle{change}
	\theorembodyfont{\normalfont}
	\theoremheaderfont{\upshape}
	\newtheorem{ejem}{Ejemplo}[section]
	\newtheorem{ejer}{Ejercicio}[section]
	\newtheorem{cejem}{Contraejemplo}[section]
}
{	% Soluciones (de los ejericicos), Observaciones y demostraciones
	\theoremstyle{nonumberplain}
	\theoremindent0cm
	\theoremheaderfont{\itshape}
	\theorembodyfont{\normalfont}
	\theoremseparator{.}
	\newtheorem{sol}{Solución}
	\newtheorem{obs}{Observación}
	\theoremsymbol{\ensuremath{\qed}}
	\newtheorem{proof}{Demostración}
}
%{	% Notas en el texto
%	\theoremstyle{nonumberplain}
%	\theoremindent0cm
%	\theorempreskip{.5em}
%	\theorempostskip{.5em}
%	\theorembodyfont{\small}
%	\theoremheaderfont{\small}
%	\newtheorem{nota}{\adforn{33}}
%}

\newcommand{\clase}[3]{
	\begin{flushright}
		\itshape Clase #1, #2 #3
	\end{flushright}
}

% Notas al pie se resetan por seccion
\makeatletter
\@addtoreset{footnote}{section}
\makeatother

% Macros (Atajos)
\newcommand{\inte}[4]{\int_{#1}^{#2} #3\, d#4}
\newcommand{\id}[1]{\mathfrak{#1}}
\newcommand{\an}[1]{\mathbb{#1}}
\newcommand{\ve}[1]{\bm{#1}}
\newcommand{\vcsp}[1]{\mathbb{#1}}
\newcommand{\V}{\mathbb{V}}
\newcommand{\F}{\mathbb{F}}
\newcommand{\Mn}[1]{\vcsp M_{n\t n}(\vcsp #1^n)}
\newcommand{\norm}{\lVert\phantom{a}\rVert}
\newcommand{\Norm}[1]{\left\lVert#1\right\rVert}
\newcommand{\iprod}{\langle\phantom{x},\phantom{x}\rangle}
\newcommand{\Iprod}[2]{\left\langle#1,#2\right\rangle}
\newcommand{\Bform}{(\phantom{a},\phantom{a})}
\newcommand{\bform}[2]{(#1,#2)}
\newcommand{\inv}{^{-1}}
\newcommand{\ort}{^\perp}
\renewcommand{\t}{\times}
\newcommand{\dt}[2]{#1_1 #2 #1_2 #2 \dots #2 #1_n}
\newcommand{\Dt}[2]{(#1_1+#2_1)^2 + \dots + (#1_n+#2_n)^2}
\newcommand{\pd}[1]{\frac{\partial}{\partial #1}}
\newcommand{\pdd}[2]{\dfrac{\partial #1}{\partial #2}}
\newcommand{\pds}[2]{\dfrac{\partial^2 #1}{\partial #2^2}} 
\newcommand{\pdx}[3]{\dfrac{\partial^2 #1}{\partial #2 \partial #3}}
\newcommand{\der}[1]{\frac{d}{d#1}}
\newcommand{\pol}[1]{#1_n x^n + #1_{n-1} x^{n-1} + \dots + #1_1 x + #1_0}
\newcommand{\Zn}{\mathbb{Z}/n\mathbb{Z}}
\newcommand{\Rn}{\mathbb{R}^n}
\newcommand{\Fn}{\mathbb{F}^n}
\newcommand{\Cn}{\mathbb{C}^n}
\newcommand{\Rm}{\mathbb{R}^m}
\newcommand{\R}{\mathbb{R}}
\newcommand{\Co}{\mathbb{C}}
\newcommand{\Z}{\mathbb{Z}}
\newcommand{\Q}{\mathbb{Q}}
\newcommand{\N}{\mathbb{N}}
\newcommand{\Ha}{\mathbb{H}}
\newcommand{\ihat}{\mathbb i}
\newcommand{\jhat}{\mathbb j}
\newcommand{\khat}{\mathbb k}
\DeclareMathOperator{\car}{car}
\DeclareMathOperator{\area}{\acute area}
\DeclareMathOperator{\traz}{tr}
\DeclareMathOperator{\gen}{gen}